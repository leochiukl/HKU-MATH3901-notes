\section{Integer Linear Programming}
\label{sect:ilp}
\begin{enumerate}
\item So far, we have focused on analyzing linear programming problems, where
the decision variables are real-valued, corresponding to ``continuous''
optimization problems. However, in practice many variables cannot be modeled as
real numbers, e.g., number of people must be integers, which bring in
``discreteness'' in optimization problems. This shows a limitation of the
linear programming framework. To handle those problems where some
variables can be integers, we need to consider another class of optimization
problems, known as \emph{integer linear programming (ILP) problems}.
\end{enumerate}
\subsection{Definition and Modelling Techniques}
\begin{enumerate}
\item \textbf{Definition.} An \defn{integer linear programming problem} can be
expressed as
\begin{align*}
\text{min/max}\quad&f(\vect{x})=\vect{c}^{T}\vect{x}=c_1x_1+\dotsb+c_nx_n \\
\text{s.t.}\quad&\vect{a}_j^{T}\vect{x}\le/=/\ge b_j \text{ for all
\(j=1,\dotsc,m\),} \\
&\vc{\text{some/all \(x_j\)'s are integers.}}
\end{align*}
\item Of course, one clear and important application of ILP is to model
linear optimization problems with some integer variables. But this is not the
only reason for studying ILP. Another more \emph{theoretical} motivation for
developing ILP is that integer variables, while being ``discrete'', turn out
to be helpful for modelling more complex ``continuous'' constraints also.
Unfortunately, there is no general ``rule'' for how this can be done, unlike
the procedure of reducing general LP problem to standard form as in
\labelcref{it:reduce-to-std-form}. Here, we will investigate one useful method
for doing this, namely through the usage of \emph{binary variables} to describe
case-dependent constraints, e.g., ``either-or constraints''.

\item \textbf{Modelling constraints through binary variables.} A \defn{binary
variable} \(x\) is one that can only takes values \(0\) or \(1\). To include a
binary variable \(x\) into an ILP problem, we can use the relationship
\(x\in\{0,1\}\) iff \(x\) is integer and \(0\le x\le 1\), where the latter two
constraints are permissible in an ILP problem. In view of this, sometimes we
just write something like ``\(x\in\{0,1\}\)'' in an ILP problem directly, which
is understood to refer to the two constraints specified above.

To illustrate the usage of binary variables for modelling constraints, consider
the following example of optimization problem involving an ``either-or''
constraint:
\begin{align*}
\text{min}\quad&-3x_1+2x_2 \\
\text{s.t.}\quad&x_1+x_2\ge 2\text{ or }x_1+2x_2\ge 3.
\end{align*}
To deal with the ``either-or'' constraint \(x_1+x_2\ge 2\text{ or }x_1+2x_2\ge
3\), we can introduce a binary variable \(y\in\{0,1\}\) and express it as
\begin{align*}
x_1+x_2&\ge 2y \\
x_1+2x_2&\ge 3(1-y) \\
y&\in\{0,1\}.
\end{align*}
The \(y\)'s on the right-hand side of the inequalities serve as a ``switch'':
If \(y=1\), then the inequality \(x_1+x_2\ge 2\) must be satisfied. If
\(y=0\), then the inequality \(x_1+2x_2\ge 3\) must be satisfied.
\item\label{it:model-disj-const} \textbf{Modelling disjunctive constraints.} In general, we
can also deal with ``either-or'' constraints involving more than two
inequalities \emph{(disjunctive constraints)}. Suppose that we are given \(m\)
constraints \(\vect{a}_{i}^{T}\vect{x}\ge b_i,\quad i=1,\dotsc,m\), with
\(\vect{a}_i\ge\vect{0}\) for every \(i\), and at least \(k\) of them need to
be satisfied. To model this disjunctive constraint, we can introduce \(m\)
binary variables \(y_1,\dotsc,y_m\) and express it as
\begin{align*}
\vect{a}_{i}^{T}\vect{x}&\ge b_iy_i\quad\forall i=1,\dotsc,m, \\
\sum_{i=1}^{m}y_i&\ge k, \\
y_i&\in\{0,1\}\quad\forall i=1,\dotsc,m
\end{align*}
Here, the \(y_i\)'s serve as both ``switches'' and also count the number of
inequalities satisfied, which is given by \(\sum_{i=1}^{m}y_i\).
\end{enumerate}
\subsection{Cutting Plane Algorithm}
\begin{enumerate}
\item After briefly studying how the ILP is useful for modelling constraints,
we are then interested in knowing how ILP problems can be \emph{solved}.
Unfortunately, this is in general a difficult task, and algorithms for solving
ILPs are often somewhat inefficient. Here, we will provide an overview of a
relatively basic algorithm for solving ILPs, known as \emph{cutting plane
algorithm}, for dealing with ILPs where all variables are required to be
integers:
\begin{align*}
\text{min}\quad&\vect{c}^{T}\vect{x} \\
\text{s.t.}\quad&A\vect{x}=\vect{b}, \\
&\vect{x}\ge\vect{0}, \\
&\text{\(\vect{x}\) integer.}
\end{align*}
\begin{note}
Here ``\(\vect{x}\) integer'' means that \(x_j\) is an integer for every
\(j=1,\dotsc,n\).
\end{note}

The cutting plane algorithm is developed based on the observation that, if the
LP problem with the integer variables constraint dropped \emph{(relaxed
problem)},
\begin{align*}
\text{min}\quad&\vect{c}^{T}\vect{x} \\
\text{s.t.}\quad&A\vect{x}=\vect{b}, \\
&\vect{x}\ge\vect{0},
\end{align*}
results in an optimal solution with all variables being integers, then such
solution would be also optimal for the ILP problem above.
\item\label{it:cut-plane-algo} \textbf{Steps in the cutting plane algorithm.}
Consider the ILP
\begin{align*}
\text{min}\quad&\vect{c}^{T}\vect{x} \\
\text{s.t.}\quad&A\vect{x}=\vect{b}, \\
&\vect{x}\ge\vect{0}, \\
&\text{\(\vect{x}\) integer.}
\end{align*}
with its LP relaxation
\begin{align*}
\text{min}\quad&\vect{c}^{T}\vect{x} \\
\text{s.t.}\quad&A\vect{x}=\vect{b}, \\
&\vect{x}\ge\vect{0}.
\end{align*}
Suppose that the LP problem has a finite optimal value.
The \defn{cutting plane algorithm} is carried out as follows.
\begin{enumerate}[label={(\arabic*)}]
\item \emph{(Optimality check)} Find an optimal solution \(\vect{x}^{*}\) to
the LP problem. If \(\vect{x}^{*}\) contains only integer variables, then
conclude that \(\vect{x}^{*}\) is optimal for the ILP problem.

\begin{note}
The \emph{dual simplex method} is often helpful for finding such optimal
solution to the LP problem.
\end{note}

\item \emph{(Finding a cutting plane)} If some entries of \(\vect{x}^{*}\) are
not integers, then find a \defn{cutting plane}
\(\{\vect{x}\in\R^{n}:\vect{a}^{T}\vect{x}\ge b\}\) that separates
\(\vect{x}^{*}\) from all integer feasible solutions, which satisfies: (i)
\(\vect{a}^{T}\vect{x}^{*}<b\) and (ii) \(\vect{a}^{T}\vect{x}\ge b\) for every
integer feasible solution to the ILP.

\begin{note}
It can be shown that such cutting plane always exists; see also
\labelcref{it:cut-plane-optim-bfs} for a special case where a ``formula'' for
cutting plane is available.
\end{note}

\item \emph{(Adding constraint to the LP problem)} Add the constraint
\(\vect{a}^{T}\vect{x}\ge b\) to the LP problem, and go back to (1).
\end{enumerate}
The basic idea is that, with the constraint \(\vect{a}^{T}\vect{x}\ge b\)
added to the LP problem, a new optimal solution to the LP problem results, and
also no integer feasible solution is lost in the process by construction of the
cutting plane. Then, intuitively, we would be able to get an integer optimal
solution ``eventually''. Here, we shall not delve into the details about the
theoretical justification of this method.
\item\label{it:cut-plane-optim-bfs} \textbf{Finding a cutting plane for optimal
basic feasible solution.} To carry out the cutting plane algorithm in
\labelcref{it:cut-plane-algo}, we would need to find a cutting plane in each
iteration, but there is not a general way to do that. Nevertheless, in the
special case where the \(\vect{x}^{*}\) found is an optimal \emph{basic
feasible solution} to the LP problem (which is the case if the dual simplex
method is used to find the optimal solution), a cutting plane is given by
\(\boxed{\sum_{i\in I_N}^{}x_i\ge 1}\) where \(I_N\) is the set of all nonbasic
indices for \(\vect{x}^{*}\) (whenever some entries in \(\vect{x}^{*}\) are not
integers).

\begin{pf}
First, we have \(\sum_{i\in I_N}^{}x_{i}^{*}=\sum_{i\in I_N}^{}0=0<1\), since
every nonbasic variable is zero by definition. Next, fix any integer feasible
solution \(\vect{x}\ge\vect{0}\). Assume to the contrary that \(\sum_{i\in
I_N}^{}x_i<1\), which implies that \(\sum_{i\in I_N}^{}x_i=0\), i.e., \(x_i=0\)
for all \(i\in I_N\). But then this means that \(B\vect{x}_{B}
=B\vect{x}_{B}+N\vect{x}_{N}=A\vect{x}=\vect{b}\), and hence
\(\vect{x}_{B}=B^{-1}\vect{b}=\vect{x}_{B}^{*}\). Therefore, we have
\(\vect{x}=\vect{x}^{*}\), suggesting that some entries in \(\vect{x}\) are not
integers, contradiction.
\end{pf}
\end{enumerate}

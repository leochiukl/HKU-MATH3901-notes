\section{Interior Point Methods}
\label{sect:int-pt-methods}
\begin{enumerate}
\item Geometrically speaking, the simplex method solves LP problems by moving
between vertices of the feasible region and improving the objective function
values at each iteration. The vertices are located at the \emph{boundary} of the
feasible region. On the other hand, for \underline{interior point} methods, we are moving
within the \emph{interior} of the feasible region to find optimal solution. Interior
point methods serve as viable alternatives to the simplex method, and often
outperforms the simplex method for ``large'' LP problems, due to the theoretical
guarantee on the computational efficiency (running in polynomial time).

In \Cref{sect:int-pt-methods}, we will study three common kinds of interior
point methods, namely (i) the \emph{affine scaling algorithm}, (ii) the
\emph{potential reduction algorithm}, and (iii) the \emph{primal path following
algorithm}. Among the three types of interior point methods, the affine
scaling algorithm should be the simplest one, so we first study this algorithm.
\end{enumerate}
\subsection{Affine Scaling Algorithm}
\begin{enumerate}
\item \textbf{Geometrical intuition.} Consider a standard form LP
\begin{align*}
\text{min}\quad&\vect{c}^{T}\vect{x} \\
\text{s.t.}\quad&A\vect{x}=\vect{b} \\
&\vect{x}\ge \vect{0}
\end{align*}
and its dual
\begin{align*}
\text{max}\quad&\vect{b}^{T}\vect{p} \\
\text{s.t.}\quad&A^{T}\vect{p}\le\vect{c} \\
\end{align*}
where \(A\in\R^{m\times n}\) and \(\vect{b}\in\R^{m}\). Let
\(P=\{\vect{x}\in\R^n:A\vect{x}=\vect{b},\vect{x}\ge\vect{0}\}\) denote the
feasible region for the primal problem. The \defn{interior} (or \defn{relative
interior}) of \(P\) is given by \(\{\vect{x}\in
P:\vect{x}>\vect{0}\}\)\footnote{Here, the ``interior'' is taken with respect
to a certain \emph{affine subspace} rather than the Euclidean space
\(\R^{n}\).}, and its elements are called interior points. Using the property
that minimizing the objective function \(\vect{c}^{T}\vect{x}\) over an
\emph{ellipsoid} is easier than doing that directly over \(P\), the affine scaling
algorithm moves between points in the interior of \(P\) by performing
minimization of \(\vect{c}^{T}\vect{x}\) over ellipsoids in the interior,
thereby reducing the objective function value at each iteration (like the
simplex method). This procedure is graphically represented as follows:
\begin{center}
\begin{tikzpicture}
\draw[blue] (0,3) -- (2,4)  -- (5,2) -- (3,0) -- (1,0.5) -- cycle;
\draw[dashed, violet] (-1.5,\fpeval{3-2.25}) -- (1.5,\fpeval{3+2.25});
\draw[-Latex, violet] (1,4.5) --node[midway, above]{\(\vect{c}\)} (1.75,4);
\draw[magenta, fill] (0,3) circle [radius=0.7mm];
\node[magenta] () at (0,3.3) {\(\vect{x}^{*}\)};
\draw[ForestGreen] (2.5,1.8) ellipse [x radius=0.5cm, y radius=0.75cm];
\draw[black, fill] (2.5,1.8) circle [radius=0.7mm];
\draw[black, fill] (2.15,2.35) circle [radius=0.7mm];
\draw[->, thick, orange] (2.5,1.8) -- (2.15,2.35);
\node[] () at (2.5,1.5) {\(\vect{x}^{0}\)};
\node[] () at (1.9,2.35) {\(\vect{x}^{1}\)};
\draw[dashed, violet, opacity=0.4] (\fpeval{2.5-1.5},\fpeval{1.8-2.25}) -- (\fpeval{2.5+1.5},\fpeval{1.8+2.25});
\draw[dashed, violet, opacity=0.4] (\fpeval{2.15-1.5},\fpeval{2.35-2.25}) -- (\fpeval{2.15+1.5},\fpeval{2.35+2.25});
\node[blue] () at (3,4) {\(P\)};
\end{tikzpicture}
\begin{tikzpicture}
\draw[blue] (0,3) -- (2,4)  -- (5,2) -- (3,0) -- (1,0.5) -- cycle;
\draw[dashed, violet] (-1.5,\fpeval{3-2.25}) -- (1.5,\fpeval{3+2.25});
\draw[-Latex, violet] (1,4.5) --node[midway, above]{\(\vect{c}\)} (1.75,4);
\draw[magenta, fill] (0,3) circle [radius=0.7mm];
\node[magenta] () at (0,3.3) {\(\vect{x}^{*}\)};
\draw[ForestGreen, opacity=0.4] (2.5,1.8) ellipse [x radius=0.5cm, y radius=0.75cm];
\draw[black, fill] (2.5,1.8) circle [radius=0.7mm];
\draw[black, fill] (2.15,2.35) circle [radius=0.7mm];
\draw[ForestGreen, rotate around={5:(2.15,2.35)}] (2.15,2.35) ellipse [x radius=0.3cm, y radius=0.6cm];
\draw[black, fill] (2,2.9) circle [radius=0.7mm];
\draw[->, thick, orange, opacity=0.4] (2.5,1.8) -- (2.15,2.35);
\draw[->, thick, orange] (2.15,2.35) -- (2,2.9);
\node[] () at (2.5,1.5) {\(\vect{x}^{0}\)};
\node[] () at (1.9,2.35) {\(\vect{x}^{1}\)};
\node[] () at (1.7,3) {\(\vect{x}^{2}\)};
\draw[dashed, violet, opacity=0.4] (\fpeval{2.15-1.5},\fpeval{2.35-2.25}) -- (\fpeval{2.15+1.5},\fpeval{2.35+2.25});
\draw[dashed, violet, opacity=0.4] (\fpeval{2-1.5},\fpeval{2.9-2.25}) -- (\fpeval{2+1.5},\fpeval{2.9+2.25});
\node[blue] () at (3,4) {\(P\)};
\end{tikzpicture}
\end{center}
We start with an interior point \(x_0\) of \(P\), and form an ellipsoid \(S_0\)
centered at \(x_0\) that is contained in the interior of \(P\). Minimizing the
objective function \(\vect{c}^{T}\vect{x}\) over the ellipsoid \(S_0\) (easy)
yields an optimal solution \(x_1\) that is in the interior of \(P\), which has a
smaller objective function value than \(x_0\). Next, we again form an
ellipsoid centered at \(x_1\) in the interior of \(P\), and the process continues.
Intuitively, after sufficiently many iterations, the resulting point would be
fairly close to the actual optimal solution \(\vect{x}^{*}\).

Equipped with some geometrical intuition about the affine scaling algorithm, we
will then study its details.
\item \textbf{Forming an ellipsoid in the interior of \(P\).} In the affine
scaling algorithm, we would need to form ellipsoids in interior, so let us
first investigate how that can be done. To form such ellipsoids, the following
result is helpful.
\begin{lemma}
\label{lma:form-ellip-interior}
Let \(\beta\in (0,1)\) be a scalar, \(\vect{y}\in\R^{n}\) with \(\vect{y}>\vect{0}\),
and let \(S:=\{\vect{x}\in\R^{n}:\sum_{i=1}^{n}(x_i-y_i)^{2}/y_i^{2}\le\beta^{2}\}\).
Then we have \(\vect{x}>\vect{0}\) for all \(\vect{x}\in S\).
\end{lemma}
\begin{pf}
Fix any \(\vect{x}\in S\). For every \(i=1,\dotsc,n\), we have \((x_i-y_i)^{2}\le\beta^{2}y_i^{2}
<y_i^{2}\), and hence \(|x_i-y_i|<y_i\). This particularly implies that \(y_i-x_i<y_i\),
and so \(x_i>0\).
\end{pf}

To see how this result helps us construct such ellipsoids, consider the
following.  Fix an interior point \(\vect{y}\) of \(P\) and let
\(Y=\diag{y_1,\dotsc,y_n}\) denote the \(n\times n\) diagonal matrix with
\((i,i)\)th entry being \(y_i\) for all \(i=1,\dotsc,n\). As
\(y_1,\dotsc,y_n>0\), \(Y\) is invertible, and hence we can express \(S\) as
\[
S=\left\{\vect{x}\in\R^{n}:\sum_{i=1}^{n}\frac{(x_i-y_i)^{2}}{y_i^{2}}\le\beta^{2}\right\}
=\{\vect{x}\in\R^{n}:(\vect{x}-\vect{y})^{T}(Y^{-1})^{2}(\vect{x}-\vect{y})\le\beta^{2}\}
=\{\vect{x}\in\R^{n}:\|Y^{-1}(\vect{x}-\vect{y})\|\le\beta\}
\]
where \(\|\cdot\|\) denotes the Euclidean norm and satisfies that
\(\|\vect{v}\|^{2}=\vect{v}^{T}\vect{v}\).  As
\((Y^{-1})^{2}=\diag{y_1^{-2},\dotsc,y_n^{-2}}\) is positive definite, \(S\) is
indeed an ellipsoid centered at \(\vect{y}\). Then, \(S_0:=S\cap
\{\vect{x}\in\R^{n}:A\vect{x}=\vect{b}\}\) would be a section of ellipsoid
centered at \(\vect{y}\) that is contained in the interior of \(P\); sometimes we also
call \(S_0\) simply as ellipsoid (with respect to \(P\)). This provides us a
method for obtaining ellipsoids in the interior of \(P\) centered at desired
points.
\item\label{it:min-over-ellip} \textbf{Minimization over ellipsoid.} After forming an ellipsoid \(S_0\)
centered at \(\vect{y}\) in the interior of \(P\), we would need to perform
minimization over the ellipsoid \(S_0\) in the affine scaling algorithm:
\begin{align*}
\text{min}\quad&\vect{c}^{T}\vect{x} \\
\text{s.t.}\quad&A\vect{x}=\vect{b} \\
&\|Y^{-1}(\vect{x}-\vect{y})\|\le\beta
\end{align*}
where \(\beta\in (0,1)\), \(\vect{y}>\vect{0}\), and \(Y=\diag{y_1,\dotsc,y_n}\).

Previously we have claimed that solving such minimization problem is “easy”.
Here we will justify this claim. By “easy”, we mean that a closed-form formula
is available for getting an optimal solution.  Before discussing about it, we
first do some preparatory work to simplify the minimization problem a bit.

Let \(\vect{d}=\vect{x}-\vect{y}\) denote the vector of “movement”. By the
feasibility of \(\vect{y}\), we have \(A\vect{y}=\vect{b}\). Also, by
construction, every \(x\in S_0\) satisfies \(A\vect{x}=\vect{b}\). These imply
that \(A\vect{d}=\vect{0}\). Now, consider the following problem that performs
minimization with respect to \(\vect{d}\) instead of \(\vect{x}\):
\begin{align*}
\text{min}\quad&\vect{c}^{T}\vect{d} \\
\text{s.t.}\quad&A\vect{d}=\vect{0} \\
&\|Y^{-1}\vect{d}\|\le\beta
\end{align*}
where \(\beta\in (0,1)\) and \(Y=\diag{y_1,\dotsc,y_n}\).

Solving this minimization problem would give us enough ingredient for solving
the minimization problem above, since for each optimal solution
\(\vect{d}^{*}\) here, a corresponding optimal solution for the minimization
problem above would be \(\vect{x}^{*}=\vect{y}+\vect{d}^{*}\). Thus we will
focus on this (simpler) problem henceforth.  The following result provides a
closed-form formula for optimal solution to this problem.

\begin{proposition}
\label{prp:min-ellip-closed-form}
Suppose that the rows of \(A\) are linearly independent and \(\vect{c}\) is not
a linear combination of the rows of \(A\). Let \(\vect{y}>\vect{0}\). Then an
optimal solution to the problem above is
\[
\vect{d}^{*}=-\beta\frac{Y^{2}(\vect{c}-A^{T}\vect{p})}{\|Y(\vect{c}-A^{T}\vect{p})\|}
\]
with \(Y=\diag{y_1,\dotsc,y_n}\),
\(\vect{p}=(AY^{2}A^{T})^{-1}AY^{2}\vect{c}\), and \(\beta\in (0,1)\).
Furthermore, \(\vect{x}^{*}=\vect{y}+\vect{d}^{*}\) is an interior point of
\(P\) and satisfies that \(\vect{c}^{T}\vect{x}^{*}=\vect{c}^{T}\vect{y}
-\beta\|Y(\vect{c}-A^{T}\vect{p})\|<\vect{c}^{T}\vect{y}\) \emph{(being in interior and
leading to decrease in objective function value after the move)}.
\end{proposition}
\begin{pf}
\textbf{Showing that \(\vect{d}^{*}\) and \(\vect{p}\) are well-defined.} First
of all, we will show the well-definedness of \(\vect{d}^{*}\) and \(\vect{p}\)
by showing that \(\|Y(\vect{c}-A^{T}\vect{p})\|> 0\) and \(AY^{2}A^{T}\) is
invertible respectively. For the former, it follows from the assumption that
\(\vect{c}\) is not a linear combination of the rows of \(A\), which implies
that \(\vect{c}-A^{T}\vect{p}\ne\vect{0}\), and hence
\(Y(\vect{c}-A^{T}\vect{p})\ne\vect{0}\). For the latter, consider the
following. Since the rows of \(A\) are linearly independent, from linear
algebra we know that \(A\) and \(A^{T}\) are both invertible.  Also, since
\(\vect{y}>\vect{0}\), we know \(y_1^{2},\dotsc,y_n^{2}>0\), and so \(Y^{2}\)
is invertible. Therefore, \(AY^{2}A^{T}\) is invertible.

\textbf{Showing the feasibility of \(\vect{d}^{*}\).} Since \(\vect{y}>0\), we
know \(Y\) is invertible, thus
\[
Y^{-1}\vect{d}^{*}=-\beta\frac{Y(\vect{c}-A^{T}\vect{p})}{\|Y(\vect{c}-A^{T}\vect{p})\|}.
\]
Taking \(\|\cdot\|\) on both sides then gives \(\|Y^{-1}\vect{d}^{*}\|=\beta\).
So for the feasibility it remains to show that \(A\vect{d}^{*}=\vect{0}\),
which is equivalent to \(AY^{2}(\vect{c}-A^{T}\vect{p})=\vect{0}\). This holds because
\(AY^{2}(\vect{c}-A^{T}\vc{\vect{p}})=AY^{2}(\vect{c}-A^{T}\vc{(AY^{2}A^{T})^{-1}AY^{2}\vect{c}})
=AY^{2}\vect{c}-AY^{2}A^{T}(AY^{2}A^{T})^{-1}AY^{2}\vect{c}=AY^{2}\vect{c}-AY^{2}\vect{c}=\vect{0}\).

\textbf{Showing the optimality of \(\vect{d}^{*}\).} For every feasible solution
\(\vect{d}\) to the problem above, we have \(A\vect{d}=\vect{0}\) and \(\|Y^{-1}\vect{d}\|\le\beta\),
and so
\begin{align*}
\vect{c}^{T}\vect{d}\overset{(A\vect{d}=\vect{0})}&{=}(\vect{c}^{T}-\vect{p}^{T}A)\vect{d}
=(\vect{c}^{T}-\vect{p}^{T}A)YY^{-1}\vect{d} \\
&=(\vc{Y(\vect{c}-A^{T}\vect{p})})^{T}\orc{Y^{-1}\vect{d}}
\underset{(\|\vect{u}\|\|\vect{v}\|\ge \vect{u}^{T}\vect{v}\ge-\|\vect{u}\|\|\vect{v}\|)}
{\overset{\text{(Cauchy-Swartz)}}{\ge}}-\|\vc{Y(\vect{c}-A^{T}\vect{p})}\|\|\orc{Y^{-1}\vect{d}}\| \\
&\ge\mgc{-\orc{\beta}\|Y(\vect{c}-A^{T}\vect{p})\|}\overset{(\vect{v}^{T}\vect{v}=\|\vect{v}\|^{2})}{=}
-\beta\frac{[Y(\vect{c}-A^{T}\vect{p})]^{T}[Y(\vect{c}-A^{T}\vect{p})]}{\|Y(\vect{c}-A^{T}\vect{p})\|} \\
&=-(\vect{c}^{T}-\vect{p}^{T}A)\beta\frac{Y^{2}(\vect{c}-A^{T}\beta)}{\|\vect{Y}(\vect{c}-A^{T}\vect{p})\|}
=(\vect{c}^{T}-\vect{p}^{T}A)\vect{d}^{*}\overset{(A\vect{d}^{*}=\vect{0})}{=}
\mgc{\vect{c}^{T}\vect{d}^{*}},
\end{align*}
which means that \(\vect{d}^{*}\) is optimal.

\textbf{Showing that the movement results in an interior point with a drop in objective function
value.} Note that \(\vect{c}^{T}\vect{x}^{*}=\vect{c}^{T}\vect{y}+\mgc{\vect{c}^{T}\vect{d}^{*}}
=\vect{c}^{T}\vect{y}\mgc{-\beta\|Y(\vect{c}-A^{T}\vect{p})\|}\). Since
\(\|Y(\vect{c}-A^{T}\vect{p})\|>0\), we have \(\vect{c}^{T}\vect{x}^{*}<\vect{c}^{T}\vect{y}\).
Also, with \(A\vect{d}^{*}=\vect{0}\) and \(\|Y^{-1}\vect{d}^{*}\|\le\beta\),
we have \(A\vect{x}^{*}=\vect{b}\) and \(\|Y^{-1}(\vect{x}^{*}-\vect{y})\|\le\beta\),
and the latter implies that \(\vect{x}^{*}>\vect{0}\) by \Cref{lma:form-ellip-interior}.
Hence, \(\vect{x}^{*}\) lies in the interior of \(P\).
\end{pf}

\begin{remark}
\item \emph{(Relationship with the iterations of the affine scaling algorithm)}
\Cref{prp:min-ellip-closed-form} demonstrates that our geometrical intuition of
“forming an ellipsoid and then minimizing over it to move to another point in
the interior” actually works. It shows that we can indeed perform such
iterations to incrementally decrease the objective function value while
remaining in the interior.
\item \emph{(Optimal value of the original problem is \(-\infty\) if
\(\vect{d}^{*}\ge\vect{0}\))} If the optimal solution obtained satisfies
\(\vect{d}^{*}\ge\vect{0}\), then we can conclude that the optimal value of the
original minimization problem with respect to \(\vect{x}\) is \(-\infty\).  To
see this, note that in such case we would have
\(\vect{x}^{*}+\alpha\vect{d}^{*}>\vect{0}\) and
\(A(\vect{x}^{*}+\alpha\vect{d}^{*})=A\vect{x}^{*}+\alpha
A\vect{d}^{*}=\vect{b}\) for all \(\alpha>0\). By
\Cref{prp:min-ellip-closed-form}, we know that
\(\vect{c}^{T}\vect{d}^{*}=\vect{c}^{T}\vect{x}^{*}-\vect{c}^{T}\vect{y}<0\),
and hence the objective function value in the original minimization problem can
get arbitrarily negative on the feasible region as we increase \(\alpha\).
\end{remark}
\item \textbf{Interpreting the vector \(\vect{p}\) as estimate for dual basic solution.}
In \Cref{prp:min-ellip-closed-form}, we have a rather complicated expression
for the vector \(\vect{p}\), but it actually carries a somewhat intuitive
interpretation, and can be viewed as a kind of “estimate” of dual basic
solution. To see this, we consider the case where the vector \(\vect{y}\) in
\Cref{prp:min-ellip-closed-form} is a nondegenerate basic feasible solution.
WLOG, we assume that the first \(m\) variables in \(\vect{y}\) are basic, so we
can write \(\vect{y}=(y_1,\dotsc,y_m,0,\dotsc,0)\), and hence \(Y=\diag{y_1,\dotsc,y_m,0,\dotsc,0}\).
Let \(Y_0:=\diag{y_1,\dotsc,y_m}\). Then we have
\(AY=\begin{bmatrix}BY_0&\vect{0}\end{bmatrix}\in\R^{m\times n}\) where
\(B\in\R^{m\times m}\)
is the corresponding basis matrix. Therefore, we can write
\begin{align*}
\vect{p}&=(AY^{2}A^{T})^{-1}AY^{2}\vect{c}=(AY(AY)^{T})^{-1}AYY\vect{c}
=(BY_0(BY_0)^{T})^{-1}BY_0^{2}\vect{c}_{B} \\
&=(B^{T})^{-1}(Y_0^{-1})^{2}B^{-1}BY_0^{2}\vect{c}_{B}
=(B^{-1})^{T}\vect{c}_{B}
\end{align*}
where \(\vect{c}_{B}\) is the vector of coefficients in the objective function
for the basic variables. Note that the final expression is the same as the
associated dual basic solution \(\vect{y}_{B}\).

But of course, the vector \(\vect{y}\) in \Cref{prp:min-ellip-closed-form}
would generally \emph{not} be a nondegenerate basic feasible solution, and so
the vector \(\vect{p}\) would \emph{not} be a dual basic solution exactly.
However, heuristically, we may treat \(\vect{p}\) as an “estimate” for dual
basic solution.

\item \textbf{Optimality condition based on duality gap.} Although the vector
\(\vect{p}\) is only an “estimate” for dual basic solution, it can still be
used to construct an (exact) optimality condition for the affine scaling
algorithm. To start with, note that under the case where \(\vect{y}\) is a
nondegenerate basic feasible solution, the vector
\(\vect{r}:=\vect{c}-A^{T}\vect{p}\) can be expressed as
\(\vect{r}=\vect{c}-A^{T}(B^{-1})^{T}\vect{c}_{B}\) which is the vector
\(\bar{\vect{c}}\) of \emph{reduced costs}. Hence, with a similar idea as
before, the vector r can be regarded as an ``estimate'' for the vector of
reduced costs.

Recall from the proof of \Cref{lma:optim-prim-feas-dual-feas} that dual
feasibility is equivalent to having nonnegative reduced costs. Although the
vector \(\vect{r}\) here only serve as an “estimated” reduced cost vector, the
same property holds for \(\vect{r}\) \emph{always}, because we have
\(\vect{r}\ge\vect{0}\iff \vect{c}-A^{T}\vect{p}\ge\vect{0}\iff
A^{T}\vect{p}\le\vect{c}\), where the last inequality corresponds to the dual
feasibility.

To derive the \emph{optimality} condition, we would need the notion of \defn{duality
gap}, which refers to the difference between primal objective function value and
dual objective function value: \(\vect{c}^{T}\vect{y}-\vect{b}^{T}\vect{p}\),
where \(\vect{y}\) and \(\vect{p}\) are primal and dual feasible solutions
respectively. By weak duality, it is always nonnegative.  Due to the primal
feasibility of \(\vect{y}\), we can write
\(\vect{c}^{T}\vect{y}-\vect{b}^{T}\vect{p}=\vect{c}^{T}\vect{y}-\vect{p}^{T}\vect{b}=\vect{c}^{T}\vect{y}-\vect{p}^{T}A\vect{y}=\vect{r}^{T}\vect{y}\),
so the duality gap is given by \(\vect{r}^{T}\vect{y}\). Also, by
\Cref{cor:primal-dual-optim}, the duality gap \(\vect{r}^{T}\vect{y}\) being
zero would imply that \(\vect{y}\) and \(\vect{p}\) are optimal solutions to
the primal and the dual respectively.  From this result, it is natural to
anticipate that under such assumptions, if the duality gap
\(\vect{r}^{T}\vect{y}\) is “small”, then \(\vect{y}\) and \(\vect{p}\)
would be “near optimal” solutions. This intuition is formalized by the
following result.
\begin{proposition}
\label{prp:near-primal-dual-optimal}
Let \(\vect{y}\) and \(\vect{p}\) be primal and dual feasible solutions
respectively with the duality gap satisfying
\(\vect{c}^{T}\vect{y}-\vect{b}^{T}\vect{p}<\varepsilon\). Then, \(\vect{y}\) and \(\vect{p}\) are
\defn{\(\varepsilon\)-optimal} for the primal and dual respectively, i.e.,
their objective function values are within the distance of \(\varepsilon\) from
the respective optimal values:
\begin{align*}
\vect{c}^{T}\vect{y}&\le \vect{c}^{T}\vect{y}<\vect{c}^{T}\vect{y}^{*}+\varepsilon, \\
\vect{b}^{T}\vect{p}^{*}-\varepsilon&<\vect{b}^{T}\vect{p}\le\vect{b}^{T}\vect{p}^{*},
\end{align*}
where \(\vect{y}^{*}\) and \(\vect{p}^{*}\) are optimal solutions for the
primal and dual respectively.
\end{proposition}
\begin{pf}
Since \(\vect{y}\) is primal feasible and \(\vect{y}^{*}\) is primal optimal,
we have \(\vect{c}^{T}\vect{y}^{*}\le\vect{c}^{T}\vect{y}\) by definition.
Also, we have \(\vect{b}^{T}\vect{p}\le\vect{c}^{T}\vect{y}\) by weak duality
and \(\vect{c}^{T}\vect{y}-\vect{b}^{T}\vect{p}<\varepsilon\) by assumption,
thus \(\vect{c}^{T}\vect{y}<\vect{b}^{T}\vect{p}+\varepsilon
\le\vect{c}^{T}\vect{y}+\varepsilon\). With a similar argument, one can show that
\(\vect{b}^{T}\vect{p}^{*}-\varepsilon<\vect{b}^{T}\vect{p}\le\vect{b}^{T}\vect{p}^{*}\).
\end{pf}
\item\label{it:affine-scal-steps} \textbf{Steps in the affine scaling
algorithm.} We now have enough ingredients to describe the \defn{affine scaling
algorithm} as follows.
\begin{enumerate}[label={(\arabic*)}]
\item \emph{(Inputs)} Specify the following inputs:
\begin{enumerate}
\item data of the standard form LP problem: \(A\), \(\vect{b}\), and
\(\vect{c}\), where we assume that the rows of \(A\) are linearly independent
and \(\vect{c}\) is not a linear combination of the rows of \(A\)
\item an initial interior point \(\vect{x}^0\) of the feasible region \(P\)
\item the optimality tolerance \(\varepsilon>0\)
\item parameter \(\beta\in (0,1)\)
\end{enumerate}
\item \emph{(Initialization)} Set \(k\leftarrow 0\).
\item \emph{(Computing estimated dual basic solution and reduced cost)} Based
on \(\vect{x}^{k}=(x_1^{k},\dotsc,x_n^{k})\), set:
\begin{itemize}
\item \(X_k\leftarrow\diag{x_1^{k},\dotsc,x_n^{k}}\)
\item \(\vect{p}^{k}\leftarrow (AX_k^{2}A^{T})^{-1}AX_{k}^{2}\vect{c}\)
\item \(\vect{r}^{k}\leftarrow \vect{c}-A^{T}\vect{p}^{k}\)
\end{itemize}
\item \emph{(Optimality check)} If \(\vect{r}^{k}\ge\vect{0}\) and
\((\vect{x}^{k})^{T}\vect{r}^{k}<\varepsilon\), then stop and conclude that
\(\vect{x}^{k}\) and \(\vect{p}^{k}\) are \(\varepsilon\)-optimal for the
primal and dual respectively.
\item \emph{(Unboundedness check)} If \(-X_k^{2}\vect{r}^{k}\ge\vect{0}\), then
stop and conclude that the optimal value is \(-\infty\).
\begin{note}
This corresponds to the case where \(\vect{d}^{*}\ge\vect{0}\) in
\Cref{prp:min-ellip-closed-form}.
\end{note}
\item \emph{(Moving to another point in the interior and repeating prior steps)}
Set \(\vect{x}^{k+1}\leftarrow \vect{x}^{k}-\beta
X_k^{2}\vect{r}^{k}/\|X_k\vect{r}^{k}\|\) and then repeat (3)-(5) but with
\(k\leftarrow k+1\).

\begin{note}
The point \(\vect{x}^{k+1}\) is an optimal solution to the following
minimization problem:
\begin{align*}
\text{min}\quad&\vect{c}^{T}\vect{x} \\
\text{s.t.}\quad&A\vect{x}=\vect{b} \\
&\|\vc{X_k}^{-1}(\vect{x}-\vc{\vect{x}^{k}})\|\le\beta
\end{align*}
Here the feasible region is a section of ellipsoid centered at \(\vect{x}^{k}\)
that is contained in the interior of \(P\).
\end{note}
\end{enumerate}
\item \textbf{Step size for the affine scaling algorithm.} For computational
efficiency, an important quantity to be determined for the affine scaling
algorithm is its step size. In each iteration, we move to from an interior
point to another by the formula \(\vect{x}^{k+1}=\vect{x}^{k}-\beta
X_k^{2}\vect{r}^{k}/\|X_k\vect{r}^{k}\|\). One clear factor that influences the
size of the movement (\emph{step size}) is the value of \(\beta\). The larger the
\(\beta\) (while being between \(0\) and \(1\)), the larger the step size;
larger step size signifies more efficient progress and thus is often preferred.
Geometrically, this phenomenon occurs because increasing the value of \(\beta\)
would enlarge the ellipsoids formed for the iterations, and hence further
movements can be obtained:
\begin{center}
\begin{tikzpicture}
\draw[blue] (0,3) -- (2,4)  -- (5,2) -- (3,0) -- (1,0.5) -- cycle;
\draw[dashed, violet] (-1.5,\fpeval{3-2.25}) -- (1.5,\fpeval{3+2.25});
\draw[-Latex, violet] (1,4.5) --node[midway, above]{\(\vect{c}\)} (1.75,4);
\draw[magenta, fill] (0,3) circle [radius=0.7mm];
\node[magenta] () at (0,3.3) {\(\vect{x}^{*}\)};
\draw[ForestGreen] (2.5,1.8) ellipse [x radius=0.5cm, y radius=0.75cm];
\draw[black, fill] (2.5,1.8) circle [radius=0.7mm];
\draw[black, fill] (2.15,2.35) circle [radius=0.7mm];
\draw[->, thick, orange] (2.5,1.8) -- (2.15,2.35);
\node[] () at (2.5,1.5) {\(\vect{x}^{0}\)};
\node[] () at (1.9,2.35) {\(\vect{x}^{1}\)};
\draw[dashed, violet, opacity=0.4] (\fpeval{2.5-1.5},\fpeval{1.8-2.25}) -- (\fpeval{2.5+1.5},\fpeval{1.8+2.25});
\draw[dashed, violet, opacity=0.4] (\fpeval{2.15-1.5},\fpeval{2.35-2.25}) -- (\fpeval{2.15+1.5},\fpeval{2.35+2.25});
\node[blue] () at (3,4) {\(P\)};
\node[] () at (2.5,-1) {Smaller \(\beta\)};
\end{tikzpicture}
\begin{tikzpicture}
\draw[blue] (0,3) -- (2,4)  -- (5,2) -- (3,0) -- (1,0.5) -- cycle;
\draw[dashed, violet] (-1.5,\fpeval{3-2.25}) -- (1.5,\fpeval{3+2.25});
\draw[-Latex, violet] (1,4.5) --node[midway, above]{\(\vect{c}\)} (1.75,4);
\draw[magenta, fill] (0,3) circle [radius=0.7mm];
\node[magenta] () at (0,3.3) {\(\vect{x}^{*}\)};
\draw[ForestGreen] (2.5,1.8) ellipse [x radius=0.8cm, y radius=1.2cm];
\draw[black, fill] (2.5,1.8) circle [radius=0.7mm];
\draw[black, fill] (1.95,2.7) circle [radius=0.7mm];
\draw[->, thick, orange] (2.5,1.8) -- (1.95,2.7);
\node[] () at (2.5,1.5) {\(\vect{x}^{0}\)};
\node[] () at (1.8,2.8) {\(\vect{x}^{1}\)};
\draw[dashed, violet, opacity=0.4] (\fpeval{2.5-1.5},\fpeval{1.8-2.25}) -- (\fpeval{2.5+1.5},\fpeval{1.8+2.25});
\draw[dashed, violet, opacity=0.4] (\fpeval{1.95-1.5},\fpeval{2.7-2.25}) -- (\fpeval{1.95+1.5},\fpeval{2.7+2.25});
\node[blue] () at (3,4) {\(P\)};
\node[] () at (2.5,-1) {Larger \(\beta\)};
\end{tikzpicture}
\end{center}
Another factor that influences the step size is the way we choose to move
between interior points.  From the pictures above, it appears that we can
indeed move \emph{beyond} the ellipsoid without exiting the interior, which can result
in an even larger step size. But to do that, we would need to change the way we
move between interior points, which leads to the discussion of \emph{long-step
variants} of the affine scaling algorithm.
\item \textbf{Long-step variants.} When we choose to move between interior
points according to \(\vect{x}^{k+1}=\vect{x}^{k}-\beta
X_k^{2}\vect{r}^{k}/\|X_k\vect{r}^{k}\|\) in the affine scaling algorithm, this
version of algorithm is said to be \defn{short-step}; as we shall see, the step
sizes for this kind of movement would be relatively small, hence the name
``short-step''. To get long-step variants, we modify the formula above by
replacing the Euclidean norm \(\|\cdot\|\) by another function that gives
smaller values. For every \(\vect{u}\in\R^{n}\), the \defn{maximum norm} (or
\defn{\(\infty\)-norm}) of \(\vect{u}\) is
\(\|\vect{u}\|_{\infty}=\max\{|u_i|:i=1,\dotsc,n\}\). It satisfies the property
that \(\|\vect{u}\|_{\infty}\le\|\vect{u}\|\) since we have
\(|u_i|=\sqrt{u_i^{2}}\le\sqrt{u_1^{2}+\dotsb+u_n^{2}}=\|\vect{u}\|\) for all
\(i=1,\dotsc,n\). So, by changing the Euclidean norm \(\|\cdot\|\) to the
maximum norm \(\|\cdot\|_{\infty}\) in the formula, we can have longer steps.
The only thing left to show is that the formula would still lead to a movement
to an \emph{interior point} and would not “overshoot”.  This is assured by the
following result.
\begin{proposition}
\label{prp:max-norm-movement-interior}
Let \(\vect{x}^{k}\) be an interior point of the feasible region \(P\), and let
\(X_k\), \(\vect{r}^{k}\), and \(\beta\) be as defined in the affine scaling
algorithm. Then \(\vect{x}^{k+1}=\vect{x}^{k}-\beta
X_k^{2}\vect{r}^{k}/\|X_k\vect{r}^{k}\|_{\infty}\) is an interior point of \(P\).
\end{proposition}
\begin{pf}
Since \(X_k^{-1}(\vect{x}^{k+1}-\vect{x}^{k})=-\beta
X_k^{2}\vect{r}^{k}/\|X_k\vect{r}^{k}\|_{\infty}\), we have
\[
\max\left\{\frac{|x_i^{k+1}-x_i^{k}|}{x_i^{k}}:i=1,\dotsc,n\right\}
=\|X_k^{-1}(\vect{x}^{k+1}-\vect{x}^{k})\|_{\infty}
=\beta\frac{\|X_k\vect{r}^{k}\|_{\infty}}{\|X_k\vect{r}^{k}\|_{\infty}}
=\beta.
\]
So, for all \(i=1,\dotsc,n\), we have \(|x_i^{k+1}-x_i^{k}|/x_i^{k}\le\beta<1\),
which implies that \(x_i^{k}-x_{i}^{k+1}<x_i^{k}\), and hence \(x_i^{k+1}>0\).

Furthermore, similar to the feasibility part of the proof of
\Cref{prp:min-ellip-closed-form}, we have \(AX_k^{2}(\vect{c}-A^{T}\vc{\vect{p}^{k}})
=AX_{k}^{2}(\vect{c}-A^{T}\vc{(AX_k^{2}A^{T})^{-1}AX_{k}^{2}\vect{c}})
=AX_{k}^{2}\vect{c}-AX_{k}^{2}A^{T}(AX_k^{2}A^{T})^{-1}AX_{k}\vect{c}=\vect{0}\),
which implies that \(A\vect{x}^{k+1}=A\vect{x}^{k}+\vect{0}=\vect{b}\). Hence,
\(\vect{x}^{k+1}\) is an interior point of \(P\).
\end{pf}

Using the maximum norm instead of the Euclidean norm is a long-step variant,
but it is not the only one. Another more popular long-step variant is to
replace the Euclidean norm \(\|\cdot\|\) by the function \(\gamma\) given by
\(\gamma(\vect{u})=\max\{u_i:u_i>0\}\) for all \(\vect{u}\in\R^{n}\). Noting
that \(\gamma(\vect{u})=\max\{u_i:u_i>0\}\le\max\{|u_i|:i=1,\dotsc,n\}
=\|\vect{u}\|_{\infty}\), this long-step variant can lead to a larger step
size, explaining why it is more popular. We can also establish a result
analogous to \Cref{prp:max-norm-movement-interior} for this long-step variant
using a similar argument.
\item \textbf{Convergence.} Based on our intuitive geometrical understanding,
the affine scaling algorithm seems to allow us to get ``very close'' to the
optimal solution after many iterations, since at each iteration we should be
``closer'' to the optimal solution. This intuition is theoretically justified
by the following result, under some regularity assumptions.

\begin{theorem}
\label{thm:affine-scal-conv}
Assume that:
\begin{enumerate}
\item The matrix \(A\) has linearly independent rows.
\item The vector \(\vect{c}\) is not a linear combination of the rows of \(A\).
\item There exists an optimal solution.
\item There exists an interior point of the feasible region.
\end{enumerate}
If we apply the long-step affine scaling algorithm with \(\varepsilon=0\),
the function \(\gamma\), and \(0<\beta<2/3\), then the sequences
\(\{\vect{x}^{k}\}\) and \(\{\vect{p}^{k}\}\) converge to some primal and dual
optimal solutions respectively.

\begin{note}
More precisely, setting ``\(\varepsilon=0\)'' means that for the optimality check
in \labelcref{it:affine-scal-steps}, we require \((\vect{x}^{k})\vect{r}^{k}=0\)
\emph{(exact optimality)}.
\end{note}
\end{theorem}
\begin{pf}
Omitted.
\end{pf}
\item \textbf{Initialization.} Apart from the theoretical guarantee on the
convergence, another important consideration in the implementation of affine
scaling algorithm is \emph{initialization}. Recall from
\labelcref{it:affine-scal-steps} that we need to start with an initial interior
point \(\vect{x}^{0}\) of the feasible region in the affine scaling algorithm.
Previously, for the simplex method, one way to
handle the initialization is to use the big-\(M\) method
(\labelcref{it:big-m-method}). Here, we apply a similar idea to find an initial
interior point.

Our original problem is the usual standard form LP problem:
\begin{align*}
\text{min}\quad&\vect{c}^{T}\vect{x} \\
\text{s.t.}\quad&A\vect{x}=\vect{b} \\
&\vect{x}\ge \vect{0}
\end{align*}
Then, consider the following LP problem that is modified from the original,
with a new variable \(x_{n+1}\) added:
\begin{align*}
\text{min}\quad&\vect{c}^{T}\vect{x}\vc{+Mx_{n+1}} \\
\text{s.t.}\quad&A\vect{x}\vc{+(\vect{b}-A\vect{1})x_{n+1}}=\vect{b} \\
&\vect{x}\ge \vect{0}, x_{n+1}\ge 0,
\end{align*}
where \(M>0\) is a ``large'' constant. For this modified problem, we can see
that \((\vect{x},x_{n+1})=(\vect{1},1)>\vect{0}\) is an interior point of the
feasible region, since we have \(A\vect{1}+(\vect{b}-A\vect{1})(1)=\vect{b}\).

Like the big-\(M\) method, as long as the original problem has an optimal
solution and \(M\) is sufficiently large, the optimal solution to this
modified problem would always satisfy \(x_{n+1}=0\), yielding an optimal
solution to the original problem.

\begin{proposition}
\label{prp:affine-scal-big-m}
If the original problem has an optimal solution, then every optimal solution
of the modified problem satisfies \(x_{n+1}=0\), provided that \(M\) is
sufficiently large.
\end{proposition}
\begin{pf}
Omitted.
\end{pf}
\item \textbf{Computational performance.} An important practical aspect about
the affine scaling algorithm is its \emph{computational performance}.  While
the affine scaling algorithm is observed to have good performance in practice,
whether the affine scaling algorithm runs in polynomial time remains an open
question. However, we can make the following observations:
\begin{enumerate}
\item \emph{(Behaviour when initializing near a vertex)}
When the affine scaling algorithm is initialized near a vertex, it has a
similar behaviour as the simplex method in the sense that it also travels near
the ``edge'' of the feasible region. In view of this, the answer to the open
question may be more likely to be negative, as we know that the simplex method
does not run in polynomial time.
\item\label{it:more-prog-far-from-bound} \emph{(More progress when being far
from boundary)} It is also observed that the affine scaling algorithm moves in
larger steps and makes more progress when the position is far from the boundary
of the feasible region (lying ``deep'' inside the feasible region). This can be
explained by noting that larger ellipsoids can be formed when the center is far
from boundary, thereby resulting in larger steps.
\end{enumerate}
\end{enumerate}
\subsection{Potential Reduction Algorithm}
\begin{enumerate}
\item \textbf{Idea.} Inspired by the observation in
\labelcref{it:more-prog-far-from-bound}, we would like to design an algorithm
that allow the iterates to get closer to the optimal solution, while making
them far from the boundary, so that more progress can be made in the
movements generally.

More specifically, consider the following standard form LP
\begin{align*}
\text{min}\quad&\vect{c}^{T}\vect{x} \\
\text{s.t.}\quad&A\vect{x}=\vect{b} \\
&\vect{x}\ge \vect{0}
\end{align*}
and its dual (with a vector \(\vect{s}\) of slack variables added)
\begin{align*}
\text{max}\quad&\vect{b}^{T}\vect{p} \\
\text{s.t.}\quad&A^{T}\vect{p}+\vect{s}=\vect{c} \\
&\vect{s}\ge\vect{0}
\end{align*}
Then, we want to achieve the following two goals in the algorithm:
\begin{enumerate}
\item Minimize the objective function value \(\vect{c}^{T}\vect{x}\) in the
standard form LP problem.
\item Keep the iterates far from boundary.
\end{enumerate}
However, these two goals are indeed conflicting, since being away from boundary
generally limits the extent of the reduction in the objective function value.
Nevertheless, these two goals can still be incorporated simultaneously using
the following \defn{potential function}:
\[
G(\vect{x},\vect{s})=q\ln (\vect{s}^{T}\vect{x})-\sum_{j=1}^{n}\ln x_j
-\sum_{j=1}^{n}\ln s_j
\]
where \(q>n\) is a constant. This leads to the \emph{potential reduction
algorithm}, which \underline{reduces} the \underline{potential} function during
the iterations. Through minimizing the potential function
\(G(\vect{x},\vect{s})\) rather than the original objective function
\(\vect{c}^{T}\vect{x}\), the iterates would still move towards an
optimal solution, but along a trajectory that is relatively far from the
boundary.
\item \textbf{Interpretation of potential function.} To better understand the
potential function, we can consider the following two parts separately:
\begin{itemize}
\item \(q\ln(\vect{s}^{T}\vect{x})\): This is related to the duality gap, by
noting that
\[
\vect{s}^{T}\vect{x}=(\vect{c}-A^{T}\vect{p})^{T}\vect{x}
=\vect{c}^{T}\vect{x}-\vect{p}^{T}\underbrace{\vect{A}\vect{x}}_{\vect{b}}
=\underbrace{\vect{c}^{T}\vect{x}-\vect{b}^{T}\vect{p}}_{\text{duality gap}}
\]
for all primal and dual feasible solutions \(\vect{x}\) and
\((\vect{p},\vect{s})\) respectively.
\item \(-\sum_{j=1}^{n}\ln x_j-\sum_{j=1}^{n}\ln s_j\): This expression would
become very positive if the primal or dual solution gets close to the boundary,
thereby \emph{penalizing} the closedness to the boundary.
\end{itemize}
Therefore, the potential function comprises of a term that capture the
closedness to optimality (through the duality gap), and a penalty term for the
closedness to the boundary. The logarithms in the penalty term amplify
the penalties when the solutions are \emph{extremely} close to the boundary,
and correspondingly the logarithm is added to the first term for
balancing the effects from the two sources.
\item \textbf{A lower bound on the potential function.} Now, we start
discussing how the potential function can help us to find an optimal solution.
The following preliminary result provides a lower bound on the potential
function, which is utilized for justifying the potential reduction theorem.
\begin{lemma}
\label{lma:poten-fn-lb}
For all \(\vect{x}>\vect{0}\) and \(\vect{s}>\vect{0}\), we have
\[
G(\vect{x},\vect{s})\ge n\ln n+(q-n)\ln(\vect{s}^{T}\vect{x})
\]
\end{lemma}
\begin{pf}
We first write
\[
G(\vect{x},\vect{s})=q\ln(\vect{s}^{T}\vect{x})
-\sum_{j=1}^{n}\ln x_j-\sum_{j=1}^{n}\ln s_j
=(q-n)\ln(\vect{s}^{T}\vect{x})
+n\ln(\vect{s}^{T}\vect{x})-\sum_{j=1}^{n}\ln x_j
-\sum_{j=1}^{n}\ln s_j.
\]
It then remains to show that \(n\ln(\vect{s}^{T}\vect{x})-\sum_{j=1}^{n}\ln x_j
-\sum_{j=1}^{n}\ln s_j\ge n\ln n\). To show this, we use the AM-GM inequality
as follows:
\begin{align*}
n\ln(\vect{s}^{T}\vect{x})-\sum_{j=1}^{n}\ln x_j -\sum_{j=1}^{n}\ln s_j
&=n\ln(\vect{s}^{T}\vect{x})-\ln\left(\prod_{j=1}^{n}x_j\right)
-\ln\left(\prod_{j=1}^{n}s_j\right) \\
&=\ln\underbrace{\left(\sum_{j=1}^{n}x_js_j\right)^{n}}
_{\mathclap{\underset{\text{(AM-GM)}}{\ge}n^{n}\prod_{j=1}^{n}x_js_j}}
-\ln\left(\prod_{j=1}^{n}x_js_{j}\right) \\
&\ge \ln\left(n^{n}\prod_{j=1}^{n}x_js_j/\prod_{j=1}^{n}x_js_j\right)
=\ln n^{n}=n\ln n.
\end{align*}
\end{pf}

\begin{note}
This result informs us that when the potential function
\(G(\vect{x},\vect{s})\) takes small values, the duality gap
\(\vect{s}^{T}\vect{x}\) would also be small. This gives us some idea on why
minimizing the potential function can lead to optimal solution still.
\end{note}
\item \textbf{Getting \(\varepsilon\)-optimal solutions by reducing potential
function iteratively.}
The following result provides a theoretical guarantee that, as long as we can
reduce the potential function by a sufficiently large amount in each iteration,
\(\varepsilon\)-optimal solutions for both the primal and dual problems can be
obtained after finitely many iterations; indeed, the result below provides an
explicit formula for the upper bound on the number of iterations needed. It
serves as the foundation for the potential reduction algorithm.

\begin{theorem}
\label{thm:poten-redu-iter-num}
Let \(\vect{x}^{0}\) and \((\vect{p}^{0},\vect{s}^{0})\) be primal and dual
feasible solutions respectively, with \(\vect{x}^{0}>\vect{0}\) and
\(\vect{s}^{0}>\vect{0}\). Let \(\varepsilon>0\) be the optimal tolerance.
Then, every algorithm that (i) maintains both the primal and dual feasibility
and (ii) reduces the potential function \(G(\vect{x},\vect{s})\) by at least a
fixed constant \(\delta>0\), in each iteration, results in solutions
\(\vect{x}^{K}\) and \((\vect{p}^{K},\vect{s}^{K})\) for both the primal and
dual problems having the duality gap
\((\vect{s}^{K})^{T}\vect{x}^{K}=\vect{c}^{T}\vect{x}^{K}
-\vect{b}^{T}\vect{p}^{K}\le\varepsilon\)\footnote{Previously, the
\(\varepsilon\)-optimality requires ``\(<\varepsilon\)'', but here we only have
``\(\le\varepsilon\)''. Nevertheless, we usually do not care much about this
subtlety, and still regard such solutions as ``\(\varepsilon\)-optimal''.} after
\[
K=\left\lceil
\frac{G(\vect{x}^{0},\vect{s}^{0})+(q-n)\ln(1/\varepsilon)-n\ln n}{\delta}
\right\rceil
\]
iterations.
\end{theorem}
\begin{pf}
By assumption the algorithm has the property that
\(G(\vect{x}^{k+1},\vect{s}^{k+1})-G(\vect{x}^{k},\vect{s}^{k})\le -\delta\) for
every \(k=0,1,2,\dotsc\). Hence after \(K\) iterations we have
\(G(\vect{x}^{K},\vect{s}^{K})\le G(\vect{x}^{0},\vect{s}^{0})-K\delta\). For
the \(K\) specified above, we then get
\[
n\ln n+\underbrace{(q-n)}_{>0}\ln((\vect{s}^{K})^{T}\vect{x}^{K})
\overset{\text{(\Cref{lma:poten-fn-lb})}}{\le}
G(\vect{x}^{K},\vect{s}^{K})\le -(q-n)\ln(1/\varepsilon)+n\ln n
=(q-n)\ln\varepsilon+n\ln n,
\]
which implies that \((\vect{s}^{K})^{T}\vect{x}^{K}\le \varepsilon\).
\end{pf}
\item\label{it:poten-redu-idea} \textbf{Idea of the potential reduction
algorithm.} \Cref{thm:poten-redu-iter-num} serves a theoretical justification
on why algorithms based on reducing the potential function work. Next, we are
going to answer an important related question: \emph{How} can such algorithm be
designed? In the following, we will explore the intuitive idea behind the
potential reduction algorithm.

In each iteration of the affine scaling algorithm, recall that we are solving
the following minimization problem:
\begin{align*}
\text{min}\quad&\vect{c}^{T}\vect{d}
=\vect{c}^{T}(\vect{x}+\vect{d})-\vect{c}^{T}\vect{x} \\
\text{s.t.}\quad&A\vect{d}=\vect{0} \\
&\|X^{-1}\vect{d}\|\le\beta
\end{align*}
where \(\beta\in(0,1)\) and \(X=\diag{x_1,\dotsc,x_n}\), with
\(\vect{x}=(x_1,\dotsc,x_n)>\vect{0}\) being the primal feasible solution in
the iteration; here the objective function is the change in the objective value
after the movement.

To adapt it for the case here, it is natural to consider the following problem:
\begin{align*}
\text{min}\quad&G(\vect{x}+\vect{d},\vect{s})-G(\vect{x},\vect{s}) \\
\text{s.t.}\quad&A\vect{d}=\vect{0} \\
&\|X^{-1}\vect{d}\|\le\beta
\end{align*}
where the objective function is the change in the potential function after the
movement. However, such objective would generally be nonlinear and so the
optimization problem is usually hard to solve. To make the problem more
manageable, we approximate the objective function
\(G(\vect{x}+\vect{d},\vect{s})-G(\vect{x},\vect{s})\) through a first-order
Taylor expansion of \(G\) at \((\vect{x},\vect{s})\):
\(G(\vect{x}+\vect{d},\vect{s})-G(\vect{x},\vect{s})
=\nabla_{\vect{x}}G(\vect{x},\vect{s})^{T}\vect{d}+O(\|\vect{d}\|^{2})\), where
\(\nabla_{\vect{x}}\) denotes the gradient containing partial derivatives with
respect to the variables in \(\vect{x}\).

After performing such approximation, we consider the following simplified
problem:
\begin{align*}
\text{min}\quad&\nabla_{\vect{x}}G(\vect{x},\vect{s})^{T}\vect{d} \\
\text{s.t.}\quad&A\vect{d}=\vect{0} \\
&\|X^{-1}\vect{d}\|\le\beta
\end{align*}
One important observation is that this problem is essentially the problem
handled in \labelcref{it:min-over-ellip}, with ``\(\vect{c}\)'' being
\(\widehat{\vect{c}}:=\nabla_{\vect{x}}G(\vect{x},\vect{s})\), and therefore
admits a closed-form solution as suggested by \Cref{prp:min-ellip-closed-form},
namely
\[
\boxed{\vect{d}^{*}=-\beta\frac{X^{2}(\widehat{\vect{c}}-A^{T}\widehat{\vect{p}})}
{\|X(\widehat{\vect{c}}-A^{T}\widehat{\vect{p}})\|}
=-\beta X\frac{\vect{u}}{\|\vect{u}\|}}
\]
with \(\widehat{\vect{p}}=(AX^{2}A^{T})^{-1}AX^{2}\widehat{\vect{c}}\)
and \(\vect{u}=X(\widehat{\vect{c}}-A^{T}\widehat{\vect{p}})\).

\item \textbf{Further expressions of \(\vect{u}\) and the change
\(G(\vect{x}+\vect{d}^{*},\vect{s})-G(\vect{x},\vect{s})\).} Since we
can explicitly derive the expression of
\(\nabla_{\vect{x}}G(\vect{x},\vect{s})\), we are able to obtain a more
explicit expression of \(\vect{u}\). Noting that
\[
\widehat{c}_i=\pdv{G(\vect{x},\vect{s})}{x_i}
=\pdv{}{x_i}\left(q\ln\sum_{i=1}^{n}s_ix_i-\sum_{i=1}^{n}\ln x_i
-\sum_{i=1}^{n}\ln s_i\right)
=\frac{qs_i}{\vect{s}^{T}\vect{x}}-\frac{1}{x_i}
\quad\text{for all \(i=1,\dotsc,n\),}
\]
we get
\((X\widehat{\vect{c}})_{i}=x_i\widehat{\vect{c}}_i
=\frac{q}{\vect{s}^{T}\vect{x}}x_is_i-1\), and hence
\(X\widehat{\vect{c}}=\frac{q}{\vect{s}^{T}\vect{x}}X\vect{s}-\vect{1}\)
with \(\vect{1}=(1,\dotsc,1)\in\R^{n}\). It follows that
\[
\vect{u}=(I-XA^{T}(AX^{2}A^{T})^{-1}AX)X\widehat{\vect{c}}
=\boxed{(I-(AX)^{T}(AX^{2}A^{T})^{-1}AX)
\left(\frac{q}{\vect{s}^{T}\vect{x}}X\vect{s}-\vect{1}\right)}.
\]
For the change \(G(\vect{x}+\vect{d}^{*},\vect{s})-G(\vect{x},\vect{s})\) in
potential function, we have
\[
G(\vect{x}+\vect{d}^{*},\vect{s})-G(\vect{x},\vect{s})
=\vc{\nabla_{\vect{x}}G(\vect{x},\vect{s})^{T}\vect{d}^{*}}
+O(\|\vect{d}^{*}\|^{2})
\overset{\text{(\Cref{prp:min-ellip-closed-form})}}{=}
=\vc{-\beta\|\vect{u}\|}+O(\beta^{2}).
\]
It can be shown that with a fixed \(\gamma\in (0,1)\), if \(\|\vect{u\|}\ge
\gamma\), then the potential function is guaranteed to be reduced by at least a
constant (depending on the value of \(\gamma\)) in each iteration, and
therefore \Cref{thm:poten-redu-iter-num} is applicable. In case
\(\|\vect{u}\|<\gamma\), updating the primal feasible solution alone would not
be enough for achieving a sufficient reduction in the potential function, and
hence one would need to update the \emph{dual} feasible solution also to
achieve that. This leads to the \emph{potential reduction algorithm}.
\item \textbf{Steps in the potential reduction algorithm.} The \defn{potential
reduction algorithm} is as follows.
\begin{enumerate}[label={(\arabic*)}]
\item \emph{(Inputs)} Specify the following inputs:
\begin{enumerate}
\item data of the standard form LP problem: \(A\), \(\vect{b}\), and
\(\vect{c}\), where \(A\) is assumed to have linearly independent rows
\item the initial primal and dual feasible solutions: \(\vect{x}^{0}\) and
\((\vect{p}^{0},\vect{s}^{0})\), with \(\vect{x}^{0}>\vect{0}\) and
\(\vect{s}^{0}>\vect{0}\)
\item the optimality tolerance \(\varepsilon>0\)
\item the parameters \(\beta\in (0,1)\), \(\gamma\in (0,1)\), and \(q>n\)
\end{enumerate}
\item \emph{(Initialization)} Set \(k\leftarrow 0\).
\item \emph{(Optimality check)} If
\((\vect{s}^{k})^{T}\vect{x}^{k}<\varepsilon\), then stop and conclude that
\(\vect{x}^{k}\) and \(\vect{p}^{k}\) are \(\varepsilon\)-optimal for the
primal and dual respectively.
\item \emph{(Computing update direction)} Based on
\(\vect{x}^{k}=(x_1^{k},\dotsc,x_n^{k})\), set:
\begin{itemize}
\item \(X_k\leftarrow\diag{x_1^{k},\dotsc,x_n^{k}}\)
\item \(\bar{A}^{k}\leftarrow (AX_k)^{T}(AX_k^{2}A^{T})^{-1}AX_k\)
\item \(\displaystyle \vect{u}^{k}\leftarrow (I-\bar{A}^{k})\left(\frac{q}
{(\vect{s}^{k})^{T}\vect{x}^{k}}X_k\vect{s}^{k}-\vect{1}\right)\)
\item \(\vect{d}^{k}\leftarrow -\beta X_k\vect{u}^{k}/\|\vect{u}^{k}\|\)
\end{itemize}
\item \emph{(Primal step)} If \(\|\vect{u}^{k}\|\ge\gamma\), then set:
\begin{itemize}
\item \(\vect{x}^{k+1}\leftarrow \vect{x}^{k}+\vect{d}^{k}\)
\item \(\vect{s}^{k+1}\leftarrow \vect{s}^{k}\)
\item \(\vect{p}^{k+1}\leftarrow \vect{p}^{k}\)
\end{itemize}
\item \emph{(Dual step)} If \(\|\vect{u}^{k}\|<\gamma\), then set:
\begin{itemize}
\item \(\vect{x}^{k+1}\leftarrow \vect{x}^{k}\)
\item \(\displaystyle \vect{s}^{k+1}\leftarrow
\frac{(\vect{s}^{k})^{T}\vect{x}^{k}}{q}(X_k)^{-1}(\vect{u}^{k}+\vect{1})\)
\item \(\displaystyle \vect{p}^{k+1}\leftarrow \vect{p}^{k}
+(AX_k^{2}A^{T})^{-1}(AX_{k})
\left(X_k\vect{s}^{k}-\frac{(\vect{s}^{k})^{T}\vect{x}^{k}}{q}\vect{1}\right)\).
\end{itemize}
\item \emph{(Repeating prior steps)} Set \(k\leftarrow k+1\) and go back to (3).
\end{enumerate}
\begin{note}
Here we shall omit the details about the derivation of the formulas in the dual
step.
\end{note}
\end{enumerate}
\subsection{Primal Path Following Algorithm}
\begin{enumerate}
\item The last type of interior point methods to be discussed in
\Cref{sect:int-pt-methods} is known as the \emph{primal path following
algorithm}. Although it is perhaps the most complex one among the three here,
it has the most superior performance and has wide applicability in practice.
As suggested by its name, the algorithm is about \underline{following} a
\underline{path} in the \underline{primal} feasible region, in order to find an
optimal solution. Indeed, this path following algorithm is somewhat similar to
the potential reduction algorithm conceptually, as both take the closedness to
the boundary into the account and try to travel along a trajectory
(path) that is relatively far from the boundary, through introducing
a penalty term.
\item \textbf{Barrier problems.} We shall motivate the path following algorithm
by the notion of \emph{barrier problems}. Consider a standard form LP problem
\begin{align*}
\text{min}\quad&\vect{c}^{T}\vect{x} \\
\text{s.t.}\quad&A\vect{x}=\vect{b} \\
&\vect{x}\ge \vect{0}
\end{align*}
In solving this LP problem, a major difficulty arises from the inequality
constraint \(\vect{x}\ge\vect{0}\); generally, equality constraints are easier
to deal with. One method to ``get rid of'' the inequality constraint (more
precisely, ``converting'' it to some other form) is to analyze the
associated \emph{barrier problems} instead, as specified in the following.

Let \(\mu>0\). The function
\(B_{\mu}(\vect{x})=\vect{c}^{T}\vect{x}\overbrace{-\mu\sum_{j=1}^{n}\ln
x_j}^{\text{penalty term}}\) is known as the \defn{(log) barrier function},
where we conventionally define \(B_{\mu}(\vect{x}):=\infty\) if \(x_j\le 0\)
for some \(j\). Then, the family of \defn{barrier problems} is given by
\begin{align*}
\text{min}\quad&B_{\mu}(\vect{x}) \\
\text{s.t.}\quad&A\vect{x}=\vect{b}
\end{align*}
for every \(\mu>0\); each \(\mu>0\) corresponds to one barrier problem, and the
larger the \(\mu\) is, the stronger the penalty is in effect. Here, we assume
that the parameters are set such that the barrier problem has a unique optimal
solution \(\vect{x}^{*}=\vect{x}(\mu)\) for every \(\mu>0\). Note that the
optimal solution must satisfy that \(\vect{x}(\mu)>\vect{0}\), for otherwise
the objective function of the barrier problem would be \(-\infty\) at
the optimal solution \(\vect{x}(\mu)\), which is impossible. Particularly, this
implies that \(\vect{x}(\mu)\) always lies in the original feasible region
\(P\) for all \(\mu>0\).
\item\label{it:cen-path-ana-cen} \textbf{Central path and analytic center.}
Considering \(\vect{x}(\mu)\) as a function of \(\mu\) with
\(\vect{x}:(0,\infty)\to P\), varying the parameter \(\mu\) would produce a
\emph{path} traced by \(\vect{x}(\mu)\) inside the feasible region \(P\), which
is known as the \defn{central path}. It can be shown that \(\lim_{\mu\to
0}\vect{x}(\mu)\) exists and equals an optimal solution to the original LP
problem; intuitively, this happens since with a very small \(\mu\), the penalty
would be negligible and hence the optimal solution to the barrier problem would
be extremely close to the optimal solution to the original LP problem.

Further insights about the central path can be gained by inspecting the
following graphical illustration:
\begin{center}
\begin{tikzpicture}
\draw[blue] (0,3) -- (2,4)  -- (5,2) -- (3,0) -- (1,0.5) -- cycle;
\draw[dashed, violet] (-1.5,\fpeval{3-2.25}) -- (1.5,\fpeval{3+2.25});
\draw[-Latex, violet] (1,4.5) --node[midway, above]{\(\vect{c}\)} (1.75,4);
\draw[magenta, fill] (0,3) circle [radius=0.7mm];
\draw[ForestGreen] (2,1.8) .. controls (2,2.5) and (1.5,3) .. (0,3);
\draw[black, fill] (1.97,2.1) circle [radius=0.5mm];
\node[] () at (2.6,2.1) {\small{\(\vect{x}(100)\)}};
\draw[black, fill] (1.76,2.5) circle [radius=0.5mm];
\node[] () at (2.3,2.5) {\small{\(\vect{x}(10)\)}};
\draw[black, fill] (1.1,2.85) circle [radius=0.5mm];
\node[] () at (1.5,2.95) {\small{\(\vect{x}(1)\)}};
\node[magenta] () at (0,3.3) {\(\vect{x}^{*}\)};
\end{tikzpicture}
\end{center}
As we increase \(\mu\), the penalty effect is more significant, creating a
stronger force to ``push away'' the resulting optimal solution
\(\vect{x}(\mu)\) from the boundary of the feasible region, which makes
\(\vect{x}(\mu)\) farther from the actual optimal solution \(\vect{x}^{*}\),
but also being more ``deep inside'' the feasible region. Intuitively, letting
\(\mu\to\infty\) would make \(\vect{x}(\mu)\) approach to the ``center'' of the
feasible region (a point that is the ``farthest'' from the boundary). Such
point \(\lim_{\mu\to\infty}\vect{x}(\mu)\) is said to be the \defn{analytic
center} of the polyhedron \(P\). \begin{note}
There are multiple kinds of centers for a polyhedron, just like the case for
triangle, which has different types of ``center'' like centroid, incenter,
circumcenter, and orthocenter.
\end{note}

The analytic center is also the optimal solution to the ``penalty-only'' problem
\begin{align*}
\text{min}\quad&-\sum_{j=1}^{n}\ln x_j \\
\text{s.t.}\quad&A\vect{x}=\vect{b}.
\end{align*}
Intuitively, this is because as \(\mu\to\infty\), the penalty effect dominates
and the original objective function would not play a role in affecting the
resulting optimal solution anymore.
\item \textbf{Solving barrier problems with approximations.} As suggested by
\labelcref{it:cen-path-ana-cen}, one way to obtain/approximate an optimal
solution to the original LP problem is to compute \(\vect{x}(\mu)\) with
\(\mu\) being close to zero. However, barrier problem is generally
\emph{difficult} to solve \warn{} due to the nonlinearity of objective
function. Therefore, like the potential reduction algorithm, we need to first
simplify the barrier problem through some approximations.

In the case of potential reduction algorithm, we need to simplify a problem
with feasible region being an ellipsoid \emph{(nonlinear)} and we have used a
first-order Taylor approximation. On the other hand, here the feasible region
of optimization problem is only a polyhedron
\(\{\vect{x}\in\R^{n}:A\vect{x}=\vect{b}\}\) \emph{(linear)}, and only using a
first-order Taylor approximation here would not be too interesting (and also
not enough for the theoretical results to work!). In view of this, we will
consider a \emph{second-order} Taylor approximation here.

Fix any \(\vect{x}>\vect{0}\). Note that
\begin{align*}
\pdv{B_{\mu}(\vect{x})}{x_i}&=c_i-\frac{\mu}{x_i}\quad\forall i=1,\dotsc,n, \\
\pdv[order=2]{B_{\mu}(\vect{x})}{x_i}&=\frac{\mu}{x_i^{2}}
\quad\forall i=1,\dotsc,n, \\
\pdv{B_{\mu}(\vect{x})}{x_i,x_j}&=0\quad\forall i\ne j.
\end{align*}
Hence, with \emph{sufficiently small} \(\vect{d}\), by second-order
Taylor expansion we have
\[
B_{\mu}(\vect{x}+\vect{d})-B_{\mu}(\vect{x})\approx
\pdv{B_{\mu}(\vect{x})}{x_i}d_i
+\frac{1}{2}\sum_{i=1}^{n}\sum_{j=1}^{n}
\pdv{B_{\mu}(\vect{x})}{x_i,x_j}d_{i}d_{j}
=(\vect{c}^{T}-\mu\vect{1}^{T}X^{-1})\vect{d}
+\frac{1}{2}\mu\vect{d}^{T}(X^{-1})^{2}\vect{d}
\]
where \(X=\diag{x_1,\dotsc,x_n}\). Based on this approximation, we will then
consider the following minimization problem, in terms of direction \(\vect{d}\)
at \(\vect{x}\):
\begin{align*}
\text{min}\quad&(\vect{c}^{T}-\mu\vect{1}^{T}X^{-1})\vect{d}
+\frac{1}{2}\mu\vect{d}^{T}(X^{-1})^{2}\vect{d}\\
\text{s.t.}\quad&A\vect{d}=\vect{0}.
\end{align*}
\item\label{it:newton-dir-fmla} \textbf{Newton direction.} Like the case of
potential reduction algorithm, this problem admits closed-form solution, but
this time it does not directly follow from previous results; we need to do some
work on solving the problem.  More specifically, we will utilize the method of
\emph{Lagrange multiplier} (from MATH2211). First, associate a vector
\(\vect{p}\) of Lagrange multipliers to the equality constraint
\(A\vect{d}=\vect{0}\) and form the Lagrangian function
\[
L(\vect{d},\vect{p})=
\underbrace{(\vect{c}^{T}-\mu\vect{1}^{T}X^{-1})\vect{d}
+\frac{1}{2}\mu\vect{d}^{T}(X^{-1})^{2}\vect{d}}_{\text{objective}}
-\vect{p}^{T}\overbrace{A\vect{d}}^{\mathclap{\text{LHS of equality constraint}}}.
\]
Then, we need to solve the following system (with \(m+n\) equations and \(m+n\)
unknowns):
\begin{align*}
\pdv{L(\vect{d},\vect{p})}{d_j}&=0\quad\forall j=1,\dotsc,n, \\
\pdv{L(\vect{d},\vect{p})}{p_i}&=0\quad\forall i=1,\dotsc,m.
\end{align*}
More explicitly, it can be expressed in vector form as
\begin{align*}
\vect{c}-\mu X^{-1}\vect{1}+\mu(X^{-1})^{2}\vect{d}-A^{T}\vect{p}&=\vect{0}, \\
A\vect{d}&=\vect{0}.
\end{align*}
Solving this system gives the solutions
\begin{align*}
\vect{d}(\mu)&=\left(I-X^{2}A^{T}(AX^{2}A^{T})^{-1}A\right)
\left(X\vect{1}-\frac{1}{\mu}X^{2}\vect{c}\right), \\
\vect{p}(\mu)&=(AX^{2}A^{T})^{-1}A(X^{2}\vect{c}-\mu X\vect{1}).
\end{align*}
The solution \(\vect{d}(\mu)\) is called the \defn{Newton direction}.  We have
the name ``\underline{Newton} direction'' since the method described here is
indeed an instance of the \emph{Newton's method} in optimization, which is to
optimize recursively \emph{local} quadratic models based on gradient and
Hessian. Here, the second-order Taylor expansion involves gradient and Hessian,
and also such approximation only works \emph{locally} (with sufficiently small
\(\vect{d}\)). The Newton direction serves as an useful tool for avoiding
the exact computation of \(\vect{x}(\mu)\) in the path following algorithm.

\item\label{it:apx-cent-path} \textbf{Approximation of central path.} The idea
of using the Newton direction for approximating the central path (``following
that path'' approximately) is as follows.
\begin{enumerate}[label={(\arabic*)}]
\item \emph{(Initialization)} Specify \(\vect{x}^{0}>0\), \(\mu^{0}\), and
\(0<\alpha<1\),
and set \(k\leftarrow 0\).
\item \emph{(Computing Newton direction)} Using the formula in
\labelcref{it:newton-dir-fmla}, compute the Newton direction
\(\vect{d}(\mu^{k})\) at \(\vect{x}^{k}\).
\item \emph{(Updating position and parameter)} Set \(\vect{x}^{k+1}\leftarrow
\vect{x}^{k}+\vect{d}(\mu^{k})\) and \(\mu^{k+1}\leftarrow \alpha \mu^{k}\)
\emph{(reducing the parameter)}.
\item Repeat (2)-(3) with \(k\leftarrow k+1\).
\end{enumerate}
In this process, we have \(\{\mu^{k}\}\to 0\) and \(\vect{x}^{k}\) remains
close to \(\vect{x}(\mu^{k})\), allowing us to obtain fairly good
approximations to \(\lim_{\mu\to 0}\vect{x}(\mu)\) with direct computations of
\(\vect{x}(\mu)\) avoided. Graphically, it looks like the following,
which suggests the iterates \(\vect{x}^{k}\)'s would travel in a ``zigzag''
trajectory that ``follows the path''.
\begin{center}
\begin{tikzpicture}
\draw[blue] (0,3) -- (2,4)  -- (5,2) -- (3,0) -- (1,0.5) -- cycle;
\draw[dashed, violet] (-1.5,\fpeval{3-2.25}) -- (1.5,\fpeval{3+2.25});
\draw[-Latex, violet] (1,4.5) --node[midway, above]{\(\vect{c}\)} (1.75,4);
\draw[magenta, fill] (0,3) circle [radius=0.7mm];
\draw[ForestGreen] (2,1.8) .. controls (2,2.5) and (1.5,3) .. (0,3);
\draw[black, fill] (1.97,2.1) circle [radius=0.5mm];
\node[] () at (2.6,2.1) {\small{\(\vect{x}(\mu^{k})\)}};
\draw[black, fill] (1.1,2.85) circle [radius=0.5mm];
\node[] () at (0.9,2.5) {\small{\(\vect{x}(\mu^{k+1})\)}};
\node[magenta] () at (0,3.3) {\(\vect{x}^{*}\)};

\draw[brown, fill] (1.75,2.1) circle [radius=0.5mm];
\node[brown] () at (1.5,2.1) {\small{\(\vect{x}^{k}\)}};

\draw[brown, fill] (1.3,3) circle [radius=0.5mm];
\node[brown] () at (1.9,3.1) {\small{\(\vect{x}^{k+1}\)}};

\draw[-Latex, brown] (1.75,2.1) -- (1.3,2.95);
\end{tikzpicture}
\end{center}
\item \textbf{Steps in the primal path following algorithm.} Building upon this
idea of approximating the central path by the Newton direction, we are now
ready to describe the \defn{primal path following algorithm} as follows.
\begin{enumerate}[label={(\arabic*)}]
\item \emph{(Inputs)} Specify the following inputs:
\begin{enumerate}
\item data of the standard form LP problem: \(A\), \(\vect{b}\), and
\(\vect{c}\), where \(A\) is assumed to have linearly independent rows
\item the initial primal and dual feasible solutions: \(\vect{x}^{0}\) and
\((\vect{p}^{0},\vect{s}^{0})\), with \(\vect{x}^{0}>\vect{0}\) and
\(\vect{s}^{0}>\vect{0}\)
\item the optimality tolerance \(\varepsilon>0\)
\item the initial barrier parameter \(\mu^{0}\) and the parameter \(\alpha\in
(0,1)\)
\end{enumerate}
\item \emph{(Initialization)} Set \(k\leftarrow 0\).
\item \emph{(Optimality check)} If
\((\vect{s}^{k})^{T}\vect{x}^{k}<\varepsilon\),  then stop and conclude that
\(\vect{x}^{k}\) and \(\vect{p}^{k}\) are \(\varepsilon\)-optimal for the
primal and dual respectively.
\item \emph{(Computing Newton direction)} Set:
\begin{itemize}
\item \(X_k\leftarrow \diag{x_1^{k},\dotsc,x_n^{k}}\)
\item \(\mu^{k+1}\leftarrow \alpha\mu^{k}\)
\item \(\vect{p}^{*},\vect{d}^{*}\leftarrow\text{the solutions
\(\vect{p},\vect{d}\) to the following system:}\)
\begin{align*}
\mu^{k+1}(X_k^{-1})^{2}\vect{d}-A^{T}\vect{p}
&=\mu^{k+1}X_{k}^{-1}\vect{1}-\vect{c}, \\
A\vect{d}&=\vect{0}
\end{align*}
\begin{note}
When solving problems, it is perhaps more practical to substitute the numerical
values into the system and then solve it directly, rather than memorizing
\faIcon{brain} the lengthy formulas in \labelcref{it:newton-dir-fmla}.
\end{note}
\end{itemize}
\item \emph{(Updating solutions)} Set:
\begin{itemize}
\item \(\vect{x}^{k+1}\leftarrow \vect{x}^{k}+\vect{d}^{*}\)
\item \(\vect{p}^{k+1}\leftarrow \vect{p}^{*}\)
\item \(\vect{s}^{k+1}\leftarrow \vect{c}-A^{T}\vect{p}^{*}\)
\end{itemize}
\item \emph{(Repeating prior steps)} Set \(k\leftarrow k+1\) and go back to
(3).
\end{enumerate}
\begin{remark}
\item \emph{(Number of iterations needed)} The number of iterations taken for
getting \(\varepsilon\)-optimal solutions is related to the choices of the
inputs; see \textcite[Theorem 9.7]{bertsimas1997introduction}.
\item \emph{(Performance)} It can be shown that this algorithm does run in
polynomial time; see \textcite{bertsimas1997introduction} for more details.
\item \emph{(Primal-dual path following algorithm)} The word ``primal'' in the
``primal path following algorithm'' indicates that the algorithm focuses on
updating the solutions for the \emph{primal} problem, and approximating central
path in the \emph{primal} feasible region. An extension is available, known as
the \emph{primal-dual path following algorithm}, which do these for both the
primal and dual problems, and hence, as one may expect, exhibit an even better
performance. Hence, the primal-dual path following algorithm is the standard
interior point method used in many software due to its superior performance.
However, we shall discuss it in details here; see \textcite[Section
9.5]{bertsimas1997introduction}.
\end{remark}
\end{enumerate}

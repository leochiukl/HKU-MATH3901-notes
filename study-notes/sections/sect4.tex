\section{Sensitivity Analysis}
\label{sect:sens-analysis}
\begin{enumerate}
\item \textbf{Why sensitivity analysis?} Consider a standard form LP problem:
\begin{align*}
\text{min}\quad&\vect{c}^{T}\vect{x} \\
\text{s.t.}\quad&A\vect{x}=\vect{b} \\
&\vect{x}\ge \vect{0}
\end{align*}
Previously, we have focused on developing methods for solving it and obtaining
optimal solutions, with the parameters \(A\), \(\vect{b}\), and \(\vect{c}\)
being fixed. However, in practice, there are often changes to \(A\),
\(\vect{b}\), and \(\vect{c}\) from the updated information (e.g., some raw
materials \faIcon{cheese} become more expensive and some goods \faIcon{hamburger}
can be sold at higher prices), which makes the previously obtained optimal
solution not necessarily optimal for the new LP problem anymore.

While one can certainly solve the new LP problem from scratch again to get a
new optimal solution, this approach may not be efficient and acceptable (e.g.,
it would not be acceptable if it takes \underline{a day} to solve for the new optimal
solution from scratch, while updates occur \underline{daily}). Therefore, in
\Cref{sect:sens-analysis}, we will develop some ways to \emph{efficiently}
analyze/estimate the impacts from changing the LP problems in certain aspects,
without solving the new LP problem from scratch. This is known as
\emph{sensitivity analysis}.

For mathematical tractability, throughout we shall focus on standard form LP
problems, with the usual assumption that the rows of \(A\) are linearly
independent.
\end{enumerate}
\subsection{Local Sensitivity Analysis}
\label{subsect:local-sens-analysis}
\begin{enumerate}
\item \textbf{Basic idea.} The first type of sensitivity analysis to be
discussed is \emph{local sensitivity analysis}. As suggested by the term
``local'', it refers to the case where the changes are ``small''. The intuitive
idea of local sensitivity analysis is that the previous obtained optimal
solution should remain optimal after ``sufficiently small'' changes.

Consider a standard form LP:
\begin{align*}
\text{min}\quad&\vect{c}^{T}\vect{x} \\
\text{s.t.}\quad&A\vect{x}=\vect{b} \\
&\vect{x}\ge \vect{0}
\end{align*}
Suppose that we have obtained an optimal solution \(\vect{x}^*\) to this LP
with an optimal basis matrix \(B\). By the definition
of optimal basis matrix, we have:
\begin{itemize}
\item \emph{(feasibility)} \(B^{-1}\vect{b}\ge\vect{0}\), and \item
\emph{(optimality)}
\(\bar{\vect{c}}^{T}=\vect{c}^{T}-\vect{c}_{B}^{T}B^{-1}A\ge\vect{0}\)
(\(\vect{0}\) is a row vector), or
\(\bar{\vect{c}}=\vect{c}-A^{T}(B^{-1})^{T}\vect{c}_{B}\ge\vect{0}\)
(\(\vect{0}\) is a column vector) after taking transpose.
\end{itemize}
After having some changes, we investigate whether these two conditions are
affected. If both of them remain to hold, then the original basis matrix \(B\)
would remain optimal, preserving the optimality.  Based on this investigation,
we can then deduce how ``small'' the changes should be to ensure the
preservation of optimality. To illustrate how this approach works, we will go
through several case studies on the possible changes to the LP problem in the
following.

\item \textbf{Case Study 1: Changes in the vector \(\vect{c}\).} Here we
consider the special case where we change \(c_j\to c_j+\delta\) for a certain
\(j=1,\dotsc,n\) only. Because there is no change on \(A\) and \(\vect{b}\),
the feasibility condition \(B^{-1}\vect{b}\ge\vect{0}\) always remains valid.
So, for the basis matrix \(B\) to remain optimal, we just need
\(\bar{\vect{c}}'=\vect{c}'-A^{T}(B^{-1})^{T}\vect{c}_{B}'\ge\vect{0}\). The situation
can be divided into two cases, and we will analyze (i) the condition for
preserving optimality and (ii) change in optimal value (assuming the condition
for preserving optimality is satisfied):
\begin{itemize}
\item \emph{Case 1: \(x_i\) is a nonbasic variable.} In this case, changing
\(c_j\to c_j+\delta\) would not affect \(\vect{c}_{B}\), so
\(\vect{c}_{B}'=\vect{c}_{B}\). Hence, \(\bar{c}_i'\ge 0\) for every \(j\ne i\).
\begin{enumerate}[label={(\arabic*)}]
\item \emph{Condition for preserving optimality:} To ensure that
\(\bar{c}_i'\ge 0\) also, we would need 
\[
\bar{c}_i'\ge 0
\iff \underbrace{c_i-\vect{A}_i^{T}(B^{-1})^{T}\vect{c}_{B}}_{\bar{c}_i}+\delta\ge 0
\iff \boxed{\delta\ge -\bar{c}_i}.
\]
\item \emph{Change in optimal value:} Since we can write
\(\vect{c}^{T}\vect{x}=\vect{c}_{B}^{T}\vect{x}_{B}\) and \(\vect{c}_{B}\) is
not affected by the change, the change in optimal value is \(\boxed{0}\).
\end{enumerate}
\item \emph{Case 2: \(x_i\) is a basic variable.} This case is more complicated
as changing \(c_i\to c_i+\delta\) would affect \(\vect{c}_{B}\). Writing
\(j=B(\ell)\) for some \(\ell\), the change \(c_i\to c_i+\delta\) would lead to
the following change: \(\vect{c}_{B}\to\vect{c}_{B}+\delta\vect{e}_{\ell}\).
\begin{enumerate}[label={(\arabic*)}]
\item \emph{Condition for preserving optimality:} Since \(x_i\) is a basic
variable, its reduced cost is always zero by \labelcref{it:reduced-cost}.
Next, for every \(j\ne i\), we can write
\begin{align*}
\bar{c}_j'&=\underbrace{c_j'}_{c_j}-\vect{A}_j^{T}(B^{-1})^{T}
\underbrace{\vect{c}_{B}'}_{\vect{c}_{B}+\delta\vect{e}_{\ell}} \\
&=\orc{c_j-\vect{A}_j^{T}(B^{-1})^{T}\vect{c}_{B}}-\delta\vc{\vect{A}_j^{T}(B^{-1})^{T}\vect{e}_{\ell}} \\
&=\orc{\bar{c}_j}-\delta\vc{\vect{e}_{\ell}^{T}B^{-1}\vect{A}_j},
\end{align*}
so \(\bar{c}_j'\ge 0\iff \boxed{\bar{c}_j\ge \delta\mgc{\vect{e}_{\ell}^{T}B^{-1}\vect{A}_j}}\),
where \(\mgc{\vect{e}_{\ell}^{T}B^{-1}\vect{A}_j}\) is the \(\ell\)th entry of
\(B^{-1}\vect{A}_j\), i.e., the intersection of the \(j\)th column and
\(\ell\)th row in the tableau from the simplex method:
\begin{center}
\begin{tabular}{cccc}
\toprule
\(-\vect{c}_{B}^{T}\vect{x}_{B}\)&\(\cdots\)&\(\bar{c}_{\ell}\)&\(\cdots\) \\
\midrule
\(\vdots\)&\(\vert\)&&\(\vert\) \\
\(x_{B(\ell)}\)&\(\cdots\)&\mgc{\(\vect{e}_{\ell}^{T}B^{-1}\vect{A}_j\)}&\(\cdots\) \\
\(\vdots\)&\(\vert\)&&\(\vert\) \\
\bottomrule
\end{tabular}
\end{center}
\item \emph{Change in optimal value:} Since we have the following change:
\(\vect{c}\to\vect{c}+\delta\vect{e}_{i}\), the change in optimal value is
\(\boxed{\delta x_i}\).
\end{enumerate}
\end{itemize}

\item \textbf{Case Study 2: Addition of a new variable.} With a new variable
\(x_{n+1}\) added, the standard form LP problem becomes
\begin{align*}
\text{min}\quad&\vect{c}^{T}\vect{x}+c_{n+1}x_{n+1} \\
\text{s.t.}\quad&A\vect{x}+\vect{A}_{n+1}x_{n+1}=\vect{b} \\
&\vect{x}\ge \vect{0}, x_{n+1}\ge 0
\end{align*}
Note that \((\vect{x},x_{n+1})=(\vect{x}^*,0)\) is a basic feasible solution to
the new LP problem with basis matrix \(B\), where \(x_{n+1}\) is a
nonbasic variable. Hence, if the basis matrix \(B\) remains optimal here, then
\((\vect{x},x_{n+1})=(\vect{x}^*,0)\) is an optimal solution to this new LP problem.

Since there are no changes to \(c_1,\dotsc,c_n\) and \(B\), we must have
\(B^{-1}\vect{b}\ge\vect{0}\) and
\(\bar{\vect{c}}=\vect{c}-A^{T}(B^{-1})^{T}\vect{c}_{B}\ge\vect{0}\). So we
only need to ensure that
\(\boxed{\bar{c}_{n+1}=c_{n+1}-\vect{c}_{B}^{T}B^{-1}\vect{A}_{n+1}^{T}\ge
0}\). When this condition is met, \((\vect{x},x_{n+1})=\boxed{(\vect{x}^*,0)}\) is
optimal, and the change in optimal value is \(\boxed{0}\) since we have
\(c_{n+1}x_{n+1}=0\) here.
\end{enumerate}
\subsection{Global Dependence on the Vectors \texorpdfstring{\(\vect{b}\) and \(\vect{c}\)}{b and c}}
\label{subsect:global-dep-bc}
\begin{enumerate}
\item After studying \emph{local} sensitivity analysis in
\Cref{subsect:local-sens-analysis}, we will analyze the dependence of the
optimal value on the vectors \(\vect{b}\) and \(\vect{c}\) from a \emph{global
perspective}.  First we discuss the global dependence on the vector \(\vect{b}\).
\item \textbf{Setup.} To emphasize the dependence of the LP problem on the
vector \(\vect{b}\), we use the notation
\(P(\vect{b}):=\{\vect{x}\in\R^n:A\vect{x}=\vect{b},\vect{x}\ge\vect{0}\}\) to
denote the feasible set of the LP problem given the vector \(\vect{b}\).
Now, let \(S:=\{\vect{b}:P(\vect{b})\text{ is nonempty}\}=\{\vect{A}\vect{x}:\vect{x}\ge\vect{0}\}\).

Then, for every \(\vect{b}\in S\), we define \(F(\vect{b}):=\min_{x\in
P(\vect{b})}\vect{c}^{T}\vect{x}\), which gives the optimal value of the LP
problem, as a function of \(\vect{b}\). By construction of \(S\), we know
\(F(\vect{b})<\infty\) always. To study the dependence of the LP problem on the
vector \(\vect{b}\), this function \(F\) on \(S\) is our main focus.

Throughout, we shall assume that the dual LP problem is feasible, i.e.,
\(P^{*}(\vect{c}):=\{\vect{y}\in\R^m:A^{T}\vect{y}\le \vect{c}\}\ne\varnothing\). In this case,
by the weak duality we know that the primal optimal value must be greater than
\(-\infty\), meaning that \(F(\vect{b})\) has to be finite for every
\(\vect{b}\in S\).
\item \textbf{Local analysis on the function \(F\).} Before studying the
function \(F\) globally, we first use the technique of local sensitivity
analysis from \Cref{subsect:local-sens-analysis} to understand better the
properties of \(F\).

First fix any \(\vect{b}^{*}\in S\). With the vector \(\vect{b}=\vect{b}^{*}\),
suppose there is a nondegenerate primal optimal basic feasible solution, having
the optimal basis matrix \(B\). Then, the vector of basic variables is given by
\(\vect{x}_{B}=B^{-1}\vect{b}^{*}>\vect{0}\), where the positivity follows from
the degeneracy. By \Cref{prp:bfs-optim-nonneg-redu-cost}, the reduced costs
satisfy \(\bar{\vect{c}}\ge\vect{0}\).

Now, suppose that we change \(\vect{b}^{*}\to \vect{b}\). As long as the
difference \(\vect{b}-\vect{b}^{*}\) is sufficiently small, \(B^{-1}\vect{b}\)
would remain componentwise positive, establishing the \emph{feasibility}. Also, this
change would not affect \(\bar{\vect{c}}\), so it remains nonnegative
componentwise, establishing the \emph{optimality}. This means that, provided
that the change is sufficiently small, \(B\) would remain as an optimal basis.
Therefore, we have \(F(\vect{b})=\vect{c}_{B}^{T}\vect{x}_{B}=
\vect{c}_{B}^{T}B^{-1}\vect{b}\), for all \(\vect{b}\) ``near'' \(\vect{b}^{*}\).

Geometrically, this suggests that \(F\) is indeed a piecewise linear function
of \(\vect{b}\). To see this, note that
\(\pdv{F}{\vect{b}}=(B^{-1})^{T}\vect{c}_{B}\) (by matrix calculus). So for all
\(\vect{b}\) ``around'' \(\vect{b}^{*}\), \(F\) takes a constant ``slope'' as a
function of \(\vect{b}\).  Of course, if \(\vect{b}\) is sufficiently far from
\(\vect{b}^{*}\), the optimal basis matrix may change and so does
\(\pdv{F}{\vect{b}}\), but \(F\) still takes a constant ``slope''. This shows
the piecewise linearity of \(F\) (under the assumption of the existence of such
nondegenerate primal optimal basic feasible solution always).

\begin{center}
\begin{tikzpicture}
\begin{axis}[domain=-1:6, axis lines=middle, xtick={0.5,1,3.5}, xticklabels={\(\vect{b}_1\), \(\vect{b}_2\), \(\vect{b}_3\)}, ytick=\empty, xmax=6.5, ymin=-1, ymax=10, xlabel=\(\vect{b}\), ylabel={\(F(\vect{b})\)}]
\addplot[blue, domain=-1:1]{1-x};
\addplot[blue, domain=1:2]{0.5*x-0.5};
\addplot[blue, domain=2:6]{1.5*x-2.5};
\end{axis}
\end{tikzpicture}
\end{center}
In this picture, \(\vect{b}_1\) and \(\vect{b}_3\) yield nondegenerate primal
optimal basic feasible solutions (they are candidates for ``\(\vect{b}^{*}\)''
here), while \(\vect{b}_2\) cannot serve as a candidate for
``\(\vect{b}^{*}\)'' here, since \(F\) is not linear ``around'' \(\vect{b}_2\);
this suggests that \(\vect{b}_2\) indeed yields a \emph{degenerate} primal
optimal basic feasible solution (knowing that every vector \(\vect{b}\) must yield
a primal optimal basic feasible solution, under the dual feasibility).

\item \textbf{Convexity of \(F\).} Now we start studying \(F\) globally. The
main property of \(F\) to be discussed here is its \emph{convexity} as a
function of \(\vect{b}\):

\begin{theorem}
\label{thm:F-cvx}
The function \(F\) is a convex function of \(\vect{b}\) on the set \(S\).
\end{theorem}
We shall offer two proofs here: one is based on the definition; another is
based on the duality theory.

\begin{pf}
\emph{(by definition)}
Fix any \(\vect{b}_1,\vect{b}_2\in S\) and any \(\lambda\in[0,1]\). Since the
optimal value of the standard form LP problem here is finite, it is always
achieved at some point in the feasible region. Hence, we have
\(F(\vect{b}_1)=\vect{c}^{T}\vect{x}_1\) and
\(F(\vect{b}_2)=\vect{c}^{T}\vect{x}_2\) for some \(\vect{x}_1\in
P(\vect{b}_1)\) and \(\vect{x}_2\in P(\vect{b}_2)\).

We claim that \(\vc{\lambda\vect{x}_1+(1-\lambda)\vect{x}_2}\in
P(\orc{\lambda\vect{b}_1+(1-\lambda)\vect{b}_2})\). To see this, consider:
\begin{itemize}
\item \(A(\vc{\lambda\vect{x}_1+(1-\lambda)\vect{x}_2})
=\lambda A\vect{x}_1+(1-\lambda)A\vect{x}_2
\overset{(\vect{x}_1\in P(\vect{b}_1), \vect{x}_2\in P(\vect{b}_2))}{=}\orc{\lambda\vect{b}_1+(1-\lambda)\vect{b}_2}\).
\item From \(\vect{x}_1,\vect{x}_2\ge \vect{0}\), we have \(\lambda\vect{x}_1+(1-\lambda)\vect{x}_2\ge\vect{0}\).
\end{itemize}
Therefore, we have
\begin{align*}
F(\orc{\lambda\vect{b}_1+(1-\lambda)\vect{b}_2})
&=\min_{\vect{x}\in P(\orc{\lambda\vect{b}_1+(1-\lambda)\vect{b}_2})}\vect{c}^{T}\vect{x} \\
&\le\vect{c}^{T}(\vc{\lambda\vect{x}_1+(1-\lambda)\vect{x}_2}) \\
&=\lambda\vect{c}^{T}\vect{x}_1+(1-\lambda)\vect{c}^{T}\vect{x}_2 \\
&=\lambda F(\vect{b}_{1})+(1-\lambda)F(\vect{b}_{2}),
\end{align*}
establishing the convexity.
\end{pf}

\begin{pf}
\emph{(by duality theory)}
By strong duality, we know \(F(\vect{b})=\min_{\vect{x}\in P(\vect{b})}\vect{c}^{T}\vect{x}
=\max_{\vect{y}\in P^*(\vect{c})}\vect{b}^{T}\vect{y}\), which is finite.

\textbf{Claim:} The dual feasible region
\(P^*(\vect{c})=\{\vect{y}\in\R^m:A^{T}\vect{y}\le\vect{c}\}\) has a vertex.

\begin{pf}
Since \(P(\vect{b})\) is a nonempty standard form polyhedron and the optimal
value is finite, applying the simplex method on the primal problem would yield
an optimal basic feasible solution \(\vect{x}\) with an optimal basis matrix
\(B\) at the termination. Then, as in \Cref{lma:optim-prim-feas-dual-feas}
we can construct a basic solution \(\vect{y}=(B^{-1})^{T}\vect{c}_{B}\) to the
dual problem. Since the reduced costs must be all nonnegative at the
termination of the simplex method, such basic solution \(\vect{y}\) is also
feasible for the dual problem. Hence it is a basic feasible solution, or
vertex, of \(P^{*}(\vect{c})\).
\end{pf}

Let \(\vect{y}_{1},\dotsc,\vect{y}_{N}\) be the vertices of \(P^*(\vect{c})\).
By \Cref{thm:vertex-optimal}, the finite optimal value must be achieved at a
vertex \(\vect{y}_{i}\). So we can write
\(F(\vect{b})=\max_{i=1,\dotsc,N}\vect{b}^{T}\vect{y}_i\). As
\(\vect{b}^{T}\vect{y}_i\) is linear and hence convex in \(\vect{b}\) for every
\(i=1,\dotsc,N\), we conclude by \Cref{prp:max-cvx-cvx} that \(F\) is convex in
\(\vect{b}\).
\end{pf}
\item\label{it:global-dep-c-g-piecewise-lin-concave} \textbf{Global dependence
on the vector \(\vect{c}\).} So far we have been analyzing the global
dependence of the vector \(\vect{b}\). Next, we will consider the global
dependence on the vector \(\vect{c}\) instead. Likewise, we let
\(T:=\{\vect{c}:P^*(\vect{c})\text{ is nonempty}\}\) and define a function
\(G\) on \(T\) by \(G(\vect{c}):=\min_{\vect{x}\in
P(\vect{b})}\vect{c}^{T}\vect{x} =\max_{\vect{y}\in
P^*(\vect{c})}\vect{b}^{T}\vect{y}\) for every \(\vect{c}\in T\), which always
takes finite values.

Let \(\vect{x}_1,\dotsc,\vect{x}_N\) be the vertices of the primal feasible
region
\(P(\vect{b})=\{\vect{x}\in\R^n:A\vect{x}=\vect{b},\vect{x}\ge\vect{0}\}\).
Then, using similar argument as before, we can write
\(G(\vect{c})=\min_{i=1,\dotsc,N}\vect{c}^{T}\vect{x}_i\), which is piecewise
linear \underline{concave}. (A function \(f:\R^n\to\R\) is \defn{concave} if
for all \(\vect{x},\vect{y}\in\R^n\) and all \(\lambda\in[0,1]\), we have
\(f(\lambda\vect{x}+(1-\lambda)\vect{y})\ge\lambda
f(\vect{x})+(1-\lambda)f(\vect{y})\). Note that the negative of a concave
function is convex.)

So, in some sense, the dependency of the optimal value on \(\vect{c}\)
(described by the function \(G\)) is somewhat ``opposite'' to that on
\(\vect{b}\) (described by the function \(F\)).
\end{enumerate}
\subsection{Parametric Programming}
\label{subsect:para-programming}
\begin{enumerate}
\item For the functions \(F(\vect{b})\) and \(G(\vect{c})\) introduced in
\Cref{subsect:global-dep-bc}, generally we can only have some \emph{bounds} on
their values:
\begin{itemize}
\item \emph{(lower bound)} \(F(\vect{b})\ge \vect{b}^{T}\vect{y}_i\) where
\(\vect{y}_i\) is any vertex of the dual feasible region \(P^*(\vect{c})\).
\item \emph{(upper bound)} \(G(\vect{c})\le \vect{c}^{T}\vect{x}_i\) where
\(\vect{x}_i\) is any vertex of the primal feasible region \(P(\vect{b})\).
\end{itemize}
In \Cref{subsect:para-programming}, we will study a special case where the
\emph{exact} optimal value can be efficiently determined as a function of a
\underline{parameter} \(\theta\) that drives the change on the vector
\(\vect{c}\) (hence the name \underline{parametric} programming).

\item \textbf{Setup.} Again, we start by defining a function \(g\) by
\(g(\theta):=\min_{\vect{x}\in
P(\vect{b})}(\vect{c}+\theta\vect{d})^{T}\vect{x}\) for all \(\theta\in\R\).
The function \(g\) represents the optimal value when the vector \(\vect{c}\) is
changed in a specific way, namely along a fixed direction \(\vect{d}\).  Here,
we shall assume that the primal feasible set \(P(\vect{b})\) is nonempty. Then,
we know \(g<\infty\), but it is still generally possible that
\(g(\theta)=-\infty\) for some \(\theta\in\R\).

\item \label{it:para-g-piecewise-affine-concave} \textbf{Piecewise affinity and
concavity of function \(g\).} For every \(\theta\in\R\) where \(g(\theta)\) is
finite (not \(-\infty\)), like before we can write
\(g(\theta)=\min_{i=1,\dotsc,N}(\vect{c}+\theta\vect{d})^{T}\vect{x}_i\) where
\(\vect{x}_1,\dotsc,\vect{x}_N\) are the vertices of the primal feasible region
\(P(\vect{b})\). From this, we can see that \(g\) is piecewise affine and
concave.
\item \label{it:cpt-g-proc} \textbf{Computing \(g(\theta)\) by the simplex
method.} As mentioned earlier, the primary purpose of the function \(g\) is to
allow us to efficiently compute the optimal value in respond to the change on
the vector \(\vect{c}\) along the direction \(\vect{d}\). One way for
performing such computation is through the (primal) simplex method, whose procedure is
described as follows:
\begin{enumerate}[label={(\arabic*)}]
\item Start with the tableau associated with a basic feasible solution
\(\vect{x}\) and the corresponding basis matrix \(B\); here due to the presence
of ``\(\theta\)'' in the coefficients for the objective function, the
\emph{reduced cost} row of the tableau (zeroth row) would contain some
\(\theta\)'s that are yet to be determined.

\emph{Example:}
\begin{center}
\begin{tabular}{c}
\vspace{0.7cm} \\
\(x_{4}=\) \\
\(x_5=\) \\
\end{tabular}
\begin{tabular}{cccccc}
&\(x_1\)&\(x_2\)&\(x_3\)&\(x_4\)&\(x_5\) \\
\toprule
\(0\)&\(-3+2\theta\)&\(3-\theta\)&\(1\)&\(0\)&0\\
\midrule
\(5\)&\(1\)&\(2\)&\(-3\)&\(1\)&\(0\) \\
\(7\)&\(2\)&\(1\)&\(-4\)&\(0\)&\(1\) \\
\bottomrule
\end{tabular}
\end{center}
\item Determine the bounds on \(\theta\) such that all the reduced costs are
nonnegative (so the current basic feasible solution is optimal), say \(\ell\le
\theta\le u\). Then, for every \(\theta\in [\ell,u]\), the value of
\(g(\theta)\) is given by the negative of the top-left corner in the tableau.

\emph{Example:} In the tableau above, all reduced costs are nonnegative iff
\(-3+2\theta\ge 0\) and \(3-\theta\ge 0\), i.e., \(3/2\le\theta\le 3\). Hence,
\(g(\theta)=0\) for every \(\theta\in[3/2,3]\).
\item Consider the case where \(\theta\) falls outside the bounds (slightly):
\begin{itemize}
\item \emph{Case 1: \(\theta>u\) (slightly).} Choose a column with negative
reduced cost to be the pivot column (e.g., by Bland's pivoting rule).  If no
entry of \(\vect{u}\) (the pivot column except the reduced cost entry) is
positive, then we have \(g(\theta)=-\infty\) for every \(\theta>u\).

If some entry of \(\vect{u}\) is positive, choose the pivot row as usual,
i.e., to be the one with the smallest ratio \(x_{B(i)}/u_i\) over all \(i\)
with \(u_i>0\). Then, perform the usual EROs on the tableau to change the basis.


\emph{Example:} In the case here, suppose \(\theta>3\) slightly. Then we would choose
the second column to be the pivot column and the first row to be the pivot row:
\begin{center}
\begin{tabular}{c}
\vspace{0.7cm} \\
\(x_{4}=\) \\
\(x_5=\) \\
\end{tabular}
\begin{tabular}{cccccc}
&\(x_1\)&\(x_2\)&\(x_3\)&\(x_4\)&\(x_5\) \\
\toprule
\(\)&\(-3+2\theta\)&\mgc{\(3-\theta\)}&\(1\)&\(0\)&\(0\)\\
\midrule
\mgc{\(5\)}&\mgc{\(1\)}&\mgc{\(\boxed{2}\)}&\mgc{\(-3\)}&\mgc{\(1\)}&\mgc{\(0\)} \\
\(7\)&\(2\)&\mgc{\(1\)}&\(-4\)&\(0\)&\(1\) \\
\bottomrule
\end{tabular}
\end{center}
After performing the usual EROs (it is a bit tricker here due to the presence
of \(\theta\)), we get
\begin{center}
\begin{tabular}{c}
\vspace{0.7cm} \\
\(x_{4}=\) \\
\(x_5=\) \\
\end{tabular}
\begin{tabular}{cccccc}
&\(x_1\)&\(x_2\)&\(x_3\)&\(x_4\)&\(x_5\) \\
\toprule
\(-7.5+2.5\theta\)&\(-4.5+2.5\theta\)&\(0\)&\(5.5-1.5\theta\)&\(-1.5+0.5\theta\)&\(0\)\\
\midrule
\(2.5\)&\(0.5\)&\(1\)&\(-1.5\)&\(0.5\)&\(0\) \\
\(4.5\)&\(1.5\)&\(0\)&\(-2.5\)&\(-0.5\)&\(1\) \\
\bottomrule
\end{tabular}
\end{center}
\item \emph{Case 2: \(\theta<\ell\) (slightly).} Choose a column with negative
reduced cost to be the pivot column (e.g., by Bland's pivoting rule).  If no
entry of \(\vect{u}\) is positive, then we have \(g(\theta)=-\infty\) for every
\(\theta<\ell\).

If some entry of \(\vect{u}\) is positive, choose the pivot row as usual, and
then perform the usual EROs on the tableau to change the basis.
\end{itemize}
\item For each resulting tableau, go back to step 2 to determine the bounds on
\(\theta\) and the corresponding value of \(g(\theta)\), and then proceed to
step 3 again...

\emph{Example:} For the tableau obtained from the case with \(\theta>3\)
slightly, the reduced costs are all nonnegative iff \(3\le \theta\le 5.5/1.5\).
So we have \(g(\theta)=-(-7.5+2.5\theta)=7.5-2.5\theta\) for every \(\theta\in[3,5.5/1.5]\).
\end{enumerate}
\end{enumerate}
